\documentclass{article}

\usepackage[english]{babel}
\usepackage[a4paper,top=2cm,bottom=2cm,left=3cm,right=3cm,marginparwidth=1.75cm]{geometry}

% Useful packages
% Harvard \usepackage[backend=biber, sorting=nyt, style=authoryear-ibid]{biblatex}
\usepackage[backend=biber, sorting=nyt, style=ieee]{biblatex}
\usepackage[inkscapeformat=png]{svg}
\usepackage[outputdir=pdf]{minted}
\usepackage{graphicx}
\graphicspath{ {figures/} }
\usepackage[colorlinks=true, allcolors=blue]{hyperref}
\usepackage{parskip}
\usepackage{float}
\floatplacement{figure}{H} % forces figures to be placed at the correct location
%\usepackage{lipsum}
\usepackage{setspace}
\doublespacing
\usepackage{multicol}
\usepackage{multirow}
\usepackage{fontspec}
\usepackage{fontawesome5}

\usepackage{amsmath,amssymb}
\usepackage{csquotes}
\usepackage{mathtools}
\usepackage{blkarray, bigstrut}

\ifPDFTeX
  \usepackage[T1]{fontenc}    % Output font encoding for international characters
  \usepackage[utf8]{inputenc} % Required for inputting international characters
  \usepackage{textcomp} % provide euro and other symbols
\else % if luatex or xetex
  \usepackage{unicode-math}
  \defaultfontfeatures{Scale=MatchLowercase}
  \defaultfontfeatures[\rmfamily]{Ligatures=TeX,Scale=1}
\fi

\usepackage{jupyter}

% Document style
%\usepackage{palatino} % Use the Palatino font
%\usepackage{microtype} % Improves spacing
%\usepackage[bf,sf,center]{titlesec} % Required for modifying section titles - bold, sans-serif, centered
%\usepackage{fancyhdr} % Required for modifying headers and footers
%\fancyhead[L]{\textsf{\rightmark}} % Top left header
%\fancyhead[R]{\textsf{\leftmark}} % Top right header
%\renewcommand{\headrulewidth}{1.4pt} % Rule under the header
%\fancyfoot[C]{\textbf{\textsf{\thepage}}} % Bottom center footer
%\renewcommand{\footrulewidth}{1.4pt} % Rule under the footer
%\pagestyle{fancy} % Use the custom headers and footers throughout the document

% Appendix dictionary template: print each word on the page
% \markboth{}{} prints the first word on the page in the top left header and the last word in the top right
%\newcommand{\entry}[4]{\markboth{#1}{#1}\textbf{#1}\ {(#2)}\ \textit{#3}\ $\bullet$\ {#4}}
\newcommand{\entry}[2]{\markboth{#1}{#1}\textbf{#1}\ $\bullet$\ {#2}}


%----------------------------------------------------------------------------------------
%
% Your submission should be between 1500 and 2000 words and I would expect you to cover:
% 
% Background and explanation of the topic; this should summarise the material presented in the module and your additional reading;
%
% Details of at least one algorithm associated with the topic;
%
% Implications for future developments in cryptography;
%
% Legal, social and ethical issues associated with your chosen topic.
%
% You should include a short bibliography listed in a standard referencing style (e.g. Harvard System).
%
%----------------------------------------------------------------------------------------

%\bibliography{references}
\addbibresource{references.bib}

\title{MA7011 – Applications in Cryptography and Cryptanalysis Quantum Simulation of Cryptographic Algorithms}
\author{Stuart Kingham: ID 21014912}

\begin{document}
\doublespacing

%\begin{center}
%{\LARGE Current and emerging online cyber-attacks, threats, and criminal acts, including related digital crimes}
%\end{center} 
%\vspace*{\fill}


%\maketitle
\begin{titlepage}
  \topskip0pt
  \vspace*{\fill}
  \begin{center}
       \vspace*{1cm}

       {\LARGE Applications in Cryptography and Cryptanalysis}

       \vspace*{1cm}
       {\large \textbf{Practical Quantum Circuits that Break Cryptography}}
       
       \vspace{0.2cm}
       {\large An Investigation into the Implementation of Quantum Circuits}
            
       %\vspace{1.5cm}

       \vfill

       \textbf{Stuart Kingham: ID 2101491}

       \vfill
                        
       \vspace{0.8cm}
     
       %\includegraphics[width=0.4\textwidth]{university}

       MA7011: Applications in Cryptography and Cryptanalysis \\
       School of Computing and Digital Media\\
       London Metropolitan University\\
       May 10, 2024
            
  \end{center}
  \vspace*{\fill}
\end{titlepage}

\pagebreak

\begin{abstract}
  In this paper I look at back to Shor's original paper on the quantum
  calculation of the discrete logarithm problem and integer
  factorisation and use the IBM quantum computing SDK \emph{QISKIT} to
  implement the algorithm.
\end{abstract}

\newpage

\singlespacing 
\tableofcontents
\listoffigures
\listoftables
\doublespacing

\pagenumbering{roman}
\pagenumbering{arabic}
\newpage

\section{Introduction}

Recently in the Harvard Business Review (HBR), \textcite{Coppersmith1997} identified the challenges regarding cyber-security oversight by company
boards.   In the HBR report survey, boards are saying that they are not seeing eye-to-eye with their CISOs, with one example being that while
\enquote{65\% of board members think their organization is at risk of a material cyber-attack, only 48\% of CISOs share that view.}

\begin{figure}[!ht] % Single column figure
  %\includegraphics[width=0.95\textwidth]{statistic_id267132_annual-amount-of-financial-damage-caused-by-reported-cybercrime-in-us-2001-2022.png}\hfill
  \includesvg[width=0.95\textwidth]{Lattice-reduction}\hfill
  \caption{Latice Reduction \autocite{Wikipedia:2021}  Source: Wikipedia.}
  \label{fig:lattice-reduction}
\end{figure}



% Section 2 is an overview of the quantum computing concepts needed to understand the SDK
% and the implementation of the two circuits.
%%%%%%%%%%%%%%%%%%%%%%%%%%%%%%%%%%%%%%%%%%%%%%%%%%%%%%%%%%%%%%%%%%%%%%%%%%%%%%%%%%%%%%%%%%%%%%%%%%%%%%%%%%%%%%%%%%%%%%%%
\section{Quantum Computing Concepts}

The fundamental building block of the is a \emph{qubit}.   Qubits can be constructed as [get details of IBM cubit], trapped-ion, photonics, silicon-based, and others

Logical cubits can be made from more than one physical qubits [\ldots] to provide error correction.

\emph{Quantum gates} \href{https://en.wikipedia.org/wiki/Quantum_logic_gate}{wiki} are reversible unitary transformations on at least one qubit \ldots

The model we use for quantum computation is  an \emph{acyclic quantum circuit} or \emph{quantum gate arrays}, where this computation is a sequence of quantum gates, measurements and initialisation of qubits to known values
\href{https://en.wikipedia.org/wiki/Quantum_circuit}{wiki}

\emph{Decoherence}


?The average T1 time of a qubit is the time it takes for a qubit to decay from the excited state to the ground state. It is important because it limits the duration of meaningful programs we can run on the quantum computer.,? stated in Qiskit Textbook (IBM, 2021b) within the section titled ?Calibrating Qubits with Qiskit Pulse.?

Notation: we use Paul Dirac's bra-ket notation to 

\subsection{Quantum Algorithms}

\subsubsection{Shor}

When it was published in \citeyear{Shor:1997} there were no actual quantum computers in existence, but
Peter Shor's paper \citetitle{Shor:1997} \cite{Shor:1997},
Following on from Bernstein and Vazirani [1993] and  Simon [1994], who an oracle problem which takes
polynomial time on a quantum computer but requires exponential time on a classical computer  ushered into
practical quantum algorithms using a \emph{quantum gate array} to construct subroutines that were used
in his algorithms to of reversible modular exponentiation, quantum Fourier transforms, prime factorisation
and solving discrete algorithms.  The last two showing the practical application of the hypothetical
quantum computer to break the, now traditional, RSA \cite{Rivest:1978}, and Diffie-Hellman \cite{Diffie:1976}
\& ElGamal \cite{ElGamal:1985} public-key cryptographic schemes.

\subsubsection{Grover's Algorithm}

\subsubsection{GEECM}





% \pagebreak
% Section 3 develops the actual code that runs on IBMs systems and shows the results, 
% demonstrating the speedups and the practical limitations on the current generation of quantum computers.
%%%%%%%%%%%%%%%%%%%%%%%%%%%%%%%%%%%%%%%%%%%%%%%%%%%%%%%%%%%%%%%%%%%%%%%%%%%%%%%%%%%%%%%%%%%%%%%%%%%%%%%%%%%%%%%%%%%%%%%%
\section{Shor \& Grover Quantum Circuit Implementations}

The IBM Quantum Platform was made available to the public in May 2016 \cite{IBM:2016b} and was composed of five superconducting qubits and was housed at the IBM T.J. Watson Research Center in New York.

Potted history of platform {https://newsroom.ibm.com/2023-12-04-IBM-Debuts-Next-Generation-Quantum-Processor-IBM-Quantum-System-Two,-Extends-Roadmap-to-Advance-Era-of-Quantum-Utility}

Blah...


% Section 4 is a review of the 
% current mathematical cryptographic techniques and why they are thought to be immune to being broken by quantum computing.
\section{Post-Quantum Cryptography}

Quantum systems can only break asymmetric cryptographic ciphers that, at their core, have mathematical problems that
appear to have exponential, or worse, complexity under classical computing, but
that can be solved more efficiently using quantum circuits algorithms.  The breaking of RSA and ElGamal
schemes by Shor's algorithms is instructive, as it uses a particular class of quantum gate, the QFT, to
speed up known algorithms to factor integers and solve discrete logarithm problems.

When NIST announced \cite{NIST:2022} in \citeyear{NIST:2022} the group of four PQC specifications selected, for general
encryption this was the CRYSTALS-Kyber algorithm; selected in part for the comparatively small encryption keys that two
parties can exchange easily and its speed of operation.  And for digital signatures NIST selected the three algorithms:
CRYSTALS-Dilithium, FALCON and SPHINCS+.

Of these, three of the algorithms are based on lattice schemes, and SPHINCS+ was selected, in part, as it
was not based on lattices, but hashing, as a backup.  NIST still is accepting new algorithms and we can
expect new standards to be added as new techniques arise.

In the \citeyear{Yalamuri:2022} paper \citetitle{Yalamuri:2022}, \citeauthor{Yalamuri:2022} identify six families of crypto-systems
that have attributes that appear at present to have an assumption of being hard for anticipated
quantum computers.

In this section we give the briefest review of these families, and some reasons for there being a lack
of efficient quantum algorithms that would break these schemes.

\subsection{Lattice-based Cryptography}

Lattice schemes have well researched problems such as the \emph{Shortest Vector Problem} (SVP) and
\emph{Learning With Errors} (LWE) known to be difficult.  These problems have the attribute that
solving the average-case instance is as hard as solving for the worst case.

The problems also involve high-dimensional geometric structures for which current quantum algorithms
struggle with.  These rich mathematical structures, along with the introduction of small random
errors in LWE for example, gives weight to the belief that they are robust against attacks.

\subsection{Code-based Cryptography}

Using the knowledge that the McEliece scheme remains unbroken, schemes bases on the difficulty of
decoding randomly generated linear codes - especially Goppa codes - are seen as a promising
bases for new schemes that have smaller key sizes than the McEliece system.

The general decoding problem for linear codes is believed to be hard for both classical and
quantum computers.

\subsection{Hash-based Cryptography}

Hash based schemes have minimal security assumptions, and have shown provable security properties
in studies.  Having a long history in digital signature protocols, there are promising
candidates for PQC.

\subsection{Multivariate Cryptography}

The authors of the study point out that although many multivariate codes have been broken, many
have not.  The hardness of solving multivariate quadratic equations over finite fields is known
to be NP-hard.  There are also a variety of existent schemes - Rainbow, Unbalanced Oil and vinegar (UOV),
Hidden Field Equations (HFE), Multivariate Quadrant (MQ) problems - and this provides a rich
area of research to improve key sizes to make these codes more practical.

\subsection{Isogeny-based Cryptography}

Although computational complexity is a problem for Isogenies-based schemes, this class relies
on the difficulty of computing isogenies between elliptic curves, especially super-singular isogeny
graphs where there is significant research.

The security of these schemes are based on well studied mathematical problems, but there is
a question whether smaller devices can cope with the computational requirements.

\subsection{Symmetric-key-based Cryptography}

For AES, and other similar symmetric schemes, the vulnerability to brute force attacks using Grover's algorithm
can be mitigated by increasing the key length.  The attractiveness of using such schemes for PQC is
tempered by the fact that they do not yet allow the properties of digital signing and non-repudiation that
asymmetric schemes give.



\section{Conclusion}

\subsection{Summary of Findings}

Recap the key points discussed in the paper.

\subsection{Personal Insight}

Personally I found reading Shor's papers from the mid-1990's \cite{Shor:1994} \cite{Shor:1997} to be enlightening.  In
short order he was able to layout the salient points from a large and complex branch of science to neatly encapsulate
the mathematical methods and thinking needed to tackle the construction of a class of quantum algorithms.  The
success of his exposition is demonstrated by the brevity of Grover's paper, clocking in at 8 pages \cite{Grover:1996}.

Present your personal viewpoint developed through the research and implementation.


\subsection{Future Work}

Suggest areas for further research or potential improvements in the method.

\subsection{Final Thoughts}

Conclude with the significance of your findings for the field of cryptography.


\pagebreak

\printbibliography

\pagebreak

\appendix

\input{appendix_environment}

\singlespace
\graphicspath{ {appendix_qiskit_intro/} }
    \hypertarget{example-qiskit-code}{%
\section{Appendix B: Example QISKIT code}\label{example-qiskit-code}}

    \begin{tcolorbox}[breakable, size=fbox, boxrule=1pt, pad at break*=1mm,colback=cellbackground, colframe=cellborder]
\prompt{In}{incolor}{1}{\boxspacing}
\begin{Verbatim}[commandchars=\\\{\}]
\PY{k+kn}{from} \PY{n+nn}{qiskit\PYZus{}ibm\PYZus{}runtime} \PY{k+kn}{import} \PY{n}{QiskitRuntimeService}

\PY{c+c1}{\PYZsh{} Save an IBM Quantum account and set it as your default account.}
\PY{c+c1}{\PYZsh{} api\PYZus{}token = \PYZdq{}41636b2275416ee2ac6f950b711a8e1417d346c976243d801919532851cc25e7802f010f5d54fc039c891c944407bc3e08d541bcac8af22372eb6e501291fcb2\PYZdq{}}
\PY{c+c1}{\PYZsh{} service = QiskitRuntimeService(channel=\PYZdq{}ibm\PYZus{}quantum\PYZdq{}, token=api\PYZus{}token)}
\PY{c+c1}{\PYZsh{} QiskitRuntimeService.save\PYZus{}account(channel=\PYZdq{}ibm\PYZus{}quantum\PYZdq{}, token=api\PYZus{}token, set\PYZus{}as\PYZus{}default=True)}

\PY{c+c1}{\PYZsh{} Load saved credentials}
\PY{n}{service} \PY{o}{=} \PY{n}{QiskitRuntimeService}\PY{p}{(}\PY{p}{)}
\end{Verbatim}
\end{tcolorbox}

    \hypertarget{migrate-to-local-simulators}{%
\subsection{Migrate to local
simulators}\label{migrate-to-local-simulators}}

\href{https://docs.quantum.ibm.com/api/migration-guides/local-simulators}{IBM
Migration Guides: Local simulators}

\hypertarget{fake-backends}{%
\subsubsection{Fake backends}\label{fake-backends}}

    \begin{tcolorbox}[breakable, size=fbox, boxrule=1pt, pad at break*=1mm,colback=cellbackground, colframe=cellborder]
\prompt{In}{incolor}{2}{\boxspacing}
\begin{Verbatim}[commandchars=\\\{\}]
\PY{k+kn}{from} \PY{n+nn}{qiskit}\PY{n+nn}{.}\PY{n+nn}{circuit}\PY{n+nn}{.}\PY{n+nn}{library} \PY{k+kn}{import} \PY{n}{RealAmplitudes}
\PY{k+kn}{from} \PY{n+nn}{qiskit}\PY{n+nn}{.}\PY{n+nn}{circuit} \PY{k+kn}{import} \PY{n}{QuantumCircuit}\PY{p}{,} \PY{n}{QuantumRegister}\PY{p}{,} \PY{n}{ClassicalRegister}
\PY{k+kn}{from} \PY{n+nn}{qiskit}\PY{n+nn}{.}\PY{n+nn}{quantum\PYZus{}info} \PY{k+kn}{import} \PY{n}{SparsePauliOp}
\PY{k+kn}{from} \PY{n+nn}{qiskit}\PY{n+nn}{.}\PY{n+nn}{transpiler}\PY{n+nn}{.}\PY{n+nn}{preset\PYZus{}passmanagers} \PY{k+kn}{import} \PY{n}{generate\PYZus{}preset\PYZus{}pass\PYZus{}manager}
\PY{k+kn}{from} \PY{n+nn}{qiskit\PYZus{}ibm\PYZus{}runtime} \PY{k+kn}{import} \PY{n}{SamplerV2} \PY{k}{as} \PY{n}{Sampler}
\PY{k+kn}{from} \PY{n+nn}{qiskit\PYZus{}ibm\PYZus{}runtime}\PY{n+nn}{.}\PY{n+nn}{fake\PYZus{}provider} \PY{k+kn}{import} \PY{n}{FakeManilaV2}
 
\PY{c+c1}{\PYZsh{} Bell Circuit}
\PY{n}{qc} \PY{o}{=} \PY{n}{QuantumCircuit}\PY{p}{(}\PY{l+m+mi}{2}\PY{p}{)}
\PY{n}{qc}\PY{o}{.}\PY{n}{h}\PY{p}{(}\PY{l+m+mi}{0}\PY{p}{)}
\PY{n}{qc}\PY{o}{.}\PY{n}{cx}\PY{p}{(}\PY{l+m+mi}{0}\PY{p}{,} \PY{l+m+mi}{1}\PY{p}{)}
\PY{n}{qc}\PY{o}{.}\PY{n}{measure\PYZus{}all}\PY{p}{(}\PY{p}{)}
 
\PY{c+c1}{\PYZsh{} Run the sampler job locally using FakeManilaV2}
\PY{n}{fake\PYZus{}manila} \PY{o}{=} \PY{n}{FakeManilaV2}\PY{p}{(}\PY{p}{)}
\PY{n}{pm} \PY{o}{=} \PY{n}{generate\PYZus{}preset\PYZus{}pass\PYZus{}manager}\PY{p}{(}\PY{n}{backend}\PY{o}{=}\PY{n}{fake\PYZus{}manila}\PY{p}{,} \PY{n}{optimization\PYZus{}level}\PY{o}{=}\PY{l+m+mi}{1}\PY{p}{)}
\PY{n}{isa\PYZus{}qc} \PY{o}{=} \PY{n}{pm}\PY{o}{.}\PY{n}{run}\PY{p}{(}\PY{n}{qc}\PY{p}{)}
 
\PY{c+c1}{\PYZsh{} You can use a fixed seed to get fixed results. }
\PY{n}{options} \PY{o}{=} \PY{p}{\PYZob{}}\PY{l+s+s2}{\PYZdq{}}\PY{l+s+s2}{simulator}\PY{l+s+s2}{\PYZdq{}}\PY{p}{:} \PY{p}{\PYZob{}}\PY{l+s+s2}{\PYZdq{}}\PY{l+s+s2}{seed\PYZus{}simulator}\PY{l+s+s2}{\PYZdq{}}\PY{p}{:} \PY{l+m+mi}{42}\PY{p}{\PYZcb{}}\PY{p}{\PYZcb{}}
\PY{n}{sampler} \PY{o}{=} \PY{n}{Sampler}\PY{p}{(}\PY{n}{backend}\PY{o}{=}\PY{n}{fake\PYZus{}manila}\PY{p}{,} \PY{n}{options}\PY{o}{=}\PY{n}{options}\PY{p}{)}
 
\PY{n}{result} \PY{o}{=} \PY{n}{sampler}\PY{o}{.}\PY{n}{run}\PY{p}{(}\PY{p}{[}\PY{n}{isa\PYZus{}qc}\PY{p}{]}\PY{p}{)}\PY{o}{.}\PY{n}{result}\PY{p}{(}\PY{p}{)}
\end{Verbatim}
\end{tcolorbox}

    \hypertarget{aersimulator}{%
\subsubsection{AerSimulator}\label{aersimulator}}

    \begin{tcolorbox}[breakable, size=fbox, boxrule=1pt, pad at break*=1mm,colback=cellbackground, colframe=cellborder]
\prompt{In}{incolor}{3}{\boxspacing}
\begin{Verbatim}[commandchars=\\\{\}]
\PY{k+kn}{from} \PY{n+nn}{qiskit\PYZus{}aer} \PY{k+kn}{import} \PY{n}{AerSimulator}
\PY{k+kn}{from} \PY{n+nn}{qiskit}\PY{n+nn}{.}\PY{n+nn}{circuit}\PY{n+nn}{.}\PY{n+nn}{library} \PY{k+kn}{import} \PY{n}{RealAmplitudes}
\PY{k+kn}{from} \PY{n+nn}{qiskit}\PY{n+nn}{.}\PY{n+nn}{circuit} \PY{k+kn}{import} \PY{n}{QuantumCircuit}\PY{p}{,} \PY{n}{QuantumRegister}\PY{p}{,} \PY{n}{ClassicalRegister}
\PY{k+kn}{from} \PY{n+nn}{qiskit}\PY{n+nn}{.}\PY{n+nn}{quantum\PYZus{}info} \PY{k+kn}{import} \PY{n}{SparsePauliOp}
\PY{k+kn}{from} \PY{n+nn}{qiskit}\PY{n+nn}{.}\PY{n+nn}{transpiler}\PY{n+nn}{.}\PY{n+nn}{preset\PYZus{}passmanagers} \PY{k+kn}{import} \PY{n}{generate\PYZus{}preset\PYZus{}pass\PYZus{}manager}
\PY{k+kn}{from} \PY{n+nn}{qiskit\PYZus{}ibm\PYZus{}runtime} \PY{k+kn}{import} \PY{n}{Session}\PY{p}{,} \PY{n}{SamplerV2} \PY{k}{as} \PY{n}{Sampler}
 
\PY{c+c1}{\PYZsh{} Bell Circuit}
\PY{n}{qc} \PY{o}{=} \PY{n}{QuantumCircuit}\PY{p}{(}\PY{l+m+mi}{2}\PY{p}{)}
\PY{n}{qc}\PY{o}{.}\PY{n}{h}\PY{p}{(}\PY{l+m+mi}{0}\PY{p}{)}
\PY{n}{qc}\PY{o}{.}\PY{n}{cx}\PY{p}{(}\PY{l+m+mi}{0}\PY{p}{,} \PY{l+m+mi}{1}\PY{p}{)}
\PY{n}{qc}\PY{o}{.}\PY{n}{measure\PYZus{}all}\PY{p}{(}\PY{p}{)}
 
\PY{c+c1}{\PYZsh{} Run the sampler job locally using AerSimulator.}
\PY{c+c1}{\PYZsh{} Session syntax is supported but ignored because local mode doesn\PYZsq{}t support sessions.}
\PY{n}{aer\PYZus{}sim} \PY{o}{=} \PY{n}{AerSimulator}\PY{p}{(}\PY{p}{)}
\PY{n}{pm} \PY{o}{=} \PY{n}{generate\PYZus{}preset\PYZus{}pass\PYZus{}manager}\PY{p}{(}\PY{n}{backend}\PY{o}{=}\PY{n}{aer\PYZus{}sim}\PY{p}{,} \PY{n}{optimization\PYZus{}level}\PY{o}{=}\PY{l+m+mi}{1}\PY{p}{)}
\PY{n}{isa\PYZus{}qc} \PY{o}{=} \PY{n}{pm}\PY{o}{.}\PY{n}{run}\PY{p}{(}\PY{n}{qc}\PY{p}{)}
\PY{k}{with} \PY{n}{Session}\PY{p}{(}\PY{n}{backend}\PY{o}{=}\PY{n}{aer\PYZus{}sim}\PY{p}{)} \PY{k}{as} \PY{n}{session}\PY{p}{:}
    \PY{n}{sampler} \PY{o}{=} \PY{n}{Sampler}\PY{p}{(}\PY{n}{session}\PY{o}{=}\PY{n}{session}\PY{p}{)}
    \PY{n}{result} \PY{o}{=} \PY{n}{sampler}\PY{o}{.}\PY{n}{run}\PY{p}{(}\PY{p}{[}\PY{n}{isa\PYZus{}qc}\PY{p}{]}\PY{p}{)}\PY{o}{.}\PY{n}{result}\PY{p}{(}\PY{p}{)}
\end{Verbatim}
\end{tcolorbox}

    \begin{Verbatim}[commandchars=\\\{\}]
/Users/skingham/Library/Anaconda/envs/py312\_qiskit/lib/python3.12/site-
packages/qiskit\_ibm\_runtime/session.py:157: UserWarning: Session is not
supported in local testing mode or when using a simulator.
  warnings.warn(
    \end{Verbatim}

    \begin{tcolorbox}[breakable, size=fbox, boxrule=1pt, pad at break*=1mm,colback=cellbackground, colframe=cellborder]
\prompt{In}{incolor}{4}{\boxspacing}
\begin{Verbatim}[commandchars=\\\{\}]
\PY{k+kn}{from} \PY{n+nn}{qiskit} \PY{k+kn}{import} \PY{n}{transpile}
\PY{k+kn}{from} \PY{n+nn}{qiskit}\PY{n+nn}{.}\PY{n+nn}{circuit}\PY{n+nn}{.}\PY{n+nn}{library} \PY{k+kn}{import} \PY{n}{RealAmplitudes}
\PY{k+kn}{from} \PY{n+nn}{qiskit}\PY{n+nn}{.}\PY{n+nn}{quantum\PYZus{}info} \PY{k+kn}{import} \PY{n}{SparsePauliOp}
\PY{k+kn}{from} \PY{n+nn}{qiskit\PYZus{}aer} \PY{k+kn}{import} \PY{n}{AerSimulator}

\PY{n}{sim} \PY{o}{=} \PY{n}{AerSimulator}\PY{p}{(}\PY{p}{)}
\PY{c+c1}{\PYZsh{} \PYZhy{}\PYZhy{}\PYZhy{}\PYZhy{}\PYZhy{}\PYZhy{}\PYZhy{}\PYZhy{}\PYZhy{}\PYZhy{}\PYZhy{}\PYZhy{}\PYZhy{}\PYZhy{}\PYZhy{}\PYZhy{}\PYZhy{}\PYZhy{}\PYZhy{}\PYZhy{}\PYZhy{}\PYZhy{}\PYZhy{}\PYZhy{}\PYZhy{}\PYZhy{}}
\PY{c+c1}{\PYZsh{} Simulating using estimator}
\PY{c+c1}{\PYZsh{}\PYZhy{}\PYZhy{}\PYZhy{}\PYZhy{}\PYZhy{}\PYZhy{}\PYZhy{}\PYZhy{}\PYZhy{}\PYZhy{}\PYZhy{}\PYZhy{}\PYZhy{}\PYZhy{}\PYZhy{}\PYZhy{}\PYZhy{}\PYZhy{}\PYZhy{}\PYZhy{}\PYZhy{}\PYZhy{}\PYZhy{}\PYZhy{}\PYZhy{}\PYZhy{}\PYZhy{}}
\PY{k+kn}{from} \PY{n+nn}{qiskit\PYZus{}aer}\PY{n+nn}{.}\PY{n+nn}{primitives} \PY{k+kn}{import} \PY{n}{EstimatorV2}

\PY{n}{psi1} \PY{o}{=} \PY{n}{transpile}\PY{p}{(}\PY{n}{RealAmplitudes}\PY{p}{(}\PY{n}{num\PYZus{}qubits}\PY{o}{=}\PY{l+m+mi}{2}\PY{p}{,} \PY{n}{reps}\PY{o}{=}\PY{l+m+mi}{2}\PY{p}{)}\PY{p}{,} \PY{n}{sim}\PY{p}{,} \PY{n}{optimization\PYZus{}level}\PY{o}{=}\PY{l+m+mi}{0}\PY{p}{)}
\PY{n}{psi2} \PY{o}{=} \PY{n}{transpile}\PY{p}{(}\PY{n}{RealAmplitudes}\PY{p}{(}\PY{n}{num\PYZus{}qubits}\PY{o}{=}\PY{l+m+mi}{2}\PY{p}{,} \PY{n}{reps}\PY{o}{=}\PY{l+m+mi}{3}\PY{p}{)}\PY{p}{,} \PY{n}{sim}\PY{p}{,} \PY{n}{optimization\PYZus{}level}\PY{o}{=}\PY{l+m+mi}{0}\PY{p}{)}

\PY{n}{H1} \PY{o}{=} \PY{n}{SparsePauliOp}\PY{o}{.}\PY{n}{from\PYZus{}list}\PY{p}{(}\PY{p}{[}\PY{p}{(}\PY{l+s+s2}{\PYZdq{}}\PY{l+s+s2}{II}\PY{l+s+s2}{\PYZdq{}}\PY{p}{,} \PY{l+m+mi}{1}\PY{p}{)}\PY{p}{,} \PY{p}{(}\PY{l+s+s2}{\PYZdq{}}\PY{l+s+s2}{IZ}\PY{l+s+s2}{\PYZdq{}}\PY{p}{,} \PY{l+m+mi}{2}\PY{p}{)}\PY{p}{,} \PY{p}{(}\PY{l+s+s2}{\PYZdq{}}\PY{l+s+s2}{XI}\PY{l+s+s2}{\PYZdq{}}\PY{p}{,} \PY{l+m+mi}{3}\PY{p}{)}\PY{p}{]}\PY{p}{)}
\PY{n}{H2} \PY{o}{=} \PY{n}{SparsePauliOp}\PY{o}{.}\PY{n}{from\PYZus{}list}\PY{p}{(}\PY{p}{[}\PY{p}{(}\PY{l+s+s2}{\PYZdq{}}\PY{l+s+s2}{IZ}\PY{l+s+s2}{\PYZdq{}}\PY{p}{,} \PY{l+m+mi}{1}\PY{p}{)}\PY{p}{]}\PY{p}{)}
\PY{n}{H3} \PY{o}{=} \PY{n}{SparsePauliOp}\PY{o}{.}\PY{n}{from\PYZus{}list}\PY{p}{(}\PY{p}{[}\PY{p}{(}\PY{l+s+s2}{\PYZdq{}}\PY{l+s+s2}{ZI}\PY{l+s+s2}{\PYZdq{}}\PY{p}{,} \PY{l+m+mi}{1}\PY{p}{)}\PY{p}{,} \PY{p}{(}\PY{l+s+s2}{\PYZdq{}}\PY{l+s+s2}{ZZ}\PY{l+s+s2}{\PYZdq{}}\PY{p}{,} \PY{l+m+mi}{1}\PY{p}{)}\PY{p}{]}\PY{p}{)}

\PY{n}{theta1} \PY{o}{=} \PY{p}{[}\PY{l+m+mi}{0}\PY{p}{,} \PY{l+m+mi}{1}\PY{p}{,} \PY{l+m+mi}{1}\PY{p}{,} \PY{l+m+mi}{2}\PY{p}{,} \PY{l+m+mi}{3}\PY{p}{,} \PY{l+m+mi}{5}\PY{p}{]}
\PY{n}{theta2} \PY{o}{=} \PY{p}{[}\PY{l+m+mi}{0}\PY{p}{,} \PY{l+m+mi}{1}\PY{p}{,} \PY{l+m+mi}{1}\PY{p}{,} \PY{l+m+mi}{2}\PY{p}{,} \PY{l+m+mi}{3}\PY{p}{,} \PY{l+m+mi}{5}\PY{p}{,} \PY{l+m+mi}{8}\PY{p}{,} \PY{l+m+mi}{13}\PY{p}{]}
\PY{n}{theta3} \PY{o}{=} \PY{p}{[}\PY{l+m+mi}{1}\PY{p}{,} \PY{l+m+mi}{2}\PY{p}{,} \PY{l+m+mi}{3}\PY{p}{,} \PY{l+m+mi}{4}\PY{p}{,} \PY{l+m+mi}{5}\PY{p}{,} \PY{l+m+mi}{6}\PY{p}{]}

\PY{n}{estimator} \PY{o}{=} \PY{n}{EstimatorV2}\PY{p}{(}\PY{p}{)}

\PY{c+c1}{\PYZsh{} calculate [ [\PYZlt{}psi1(theta1)|H1|psi1(theta1)\PYZgt{},}
\PY{c+c1}{\PYZsh{}              \PYZlt{}psi1(theta3)|H3|psi1(theta3)\PYZgt{}],}
\PY{c+c1}{\PYZsh{}             [\PYZlt{}psi2(theta2)|H2|psi2(theta2)\PYZgt{}] ]}
\PY{n}{job} \PY{o}{=} \PY{n}{estimator}\PY{o}{.}\PY{n}{run}\PY{p}{(}
    \PY{p}{[}
        \PY{p}{(}\PY{n}{psi1}\PY{p}{,} \PY{p}{[}\PY{n}{H1}\PY{p}{,} \PY{n}{H3}\PY{p}{]}\PY{p}{,} \PY{p}{[}\PY{n}{theta1}\PY{p}{,} \PY{n}{theta3}\PY{p}{]}\PY{p}{)}\PY{p}{,}
        \PY{p}{(}\PY{n}{psi2}\PY{p}{,} \PY{n}{H2}\PY{p}{,} \PY{n}{theta2}\PY{p}{)}
    \PY{p}{]}\PY{p}{,}
    \PY{n}{precision}\PY{o}{=}\PY{l+m+mf}{0.01}
\PY{p}{)}
\PY{n}{result} \PY{o}{=} \PY{n}{job}\PY{o}{.}\PY{n}{result}\PY{p}{(}\PY{p}{)}
\PY{n+nb}{print}\PY{p}{(}\PY{l+s+sa}{f}\PY{l+s+s2}{\PYZdq{}}\PY{l+s+s2}{expectation values : psi1 = }\PY{l+s+si}{\PYZob{}}\PY{n}{result}\PY{p}{[}\PY{l+m+mi}{0}\PY{p}{]}\PY{o}{.}\PY{n}{data}\PY{o}{.}\PY{n}{evs}\PY{l+s+si}{\PYZcb{}}\PY{l+s+s2}{, psi2 = }\PY{l+s+si}{\PYZob{}}\PY{n}{result}\PY{p}{[}\PY{l+m+mi}{1}\PY{p}{]}\PY{o}{.}\PY{n}{data}\PY{o}{.}\PY{n}{evs}\PY{l+s+si}{\PYZcb{}}\PY{l+s+s2}{\PYZdq{}}\PY{p}{)}

\PY{c+c1}{\PYZsh{} \PYZhy{}\PYZhy{}\PYZhy{}\PYZhy{}\PYZhy{}\PYZhy{}\PYZhy{}\PYZhy{}\PYZhy{}\PYZhy{}\PYZhy{}\PYZhy{}\PYZhy{}\PYZhy{}\PYZhy{}\PYZhy{}\PYZhy{}\PYZhy{}\PYZhy{}\PYZhy{}\PYZhy{}\PYZhy{}\PYZhy{}\PYZhy{}\PYZhy{}\PYZhy{}}
\PY{c+c1}{\PYZsh{} Simulating using sampler}
\PY{c+c1}{\PYZsh{} \PYZhy{}\PYZhy{}\PYZhy{}\PYZhy{}\PYZhy{}\PYZhy{}\PYZhy{}\PYZhy{}\PYZhy{}\PYZhy{}\PYZhy{}\PYZhy{}\PYZhy{}\PYZhy{}\PYZhy{}\PYZhy{}\PYZhy{}\PYZhy{}\PYZhy{}\PYZhy{}\PYZhy{}\PYZhy{}\PYZhy{}\PYZhy{}\PYZhy{}\PYZhy{}}
\PY{k+kn}{from} \PY{n+nn}{qiskit\PYZus{}aer}\PY{n+nn}{.}\PY{n+nn}{primitives} \PY{k+kn}{import} \PY{n}{SamplerV2}
\PY{k+kn}{from} \PY{n+nn}{qiskit} \PY{k+kn}{import} \PY{n}{QuantumCircuit}

\PY{c+c1}{\PYZsh{} create a Bell circuit}
\PY{n}{bell} \PY{o}{=} \PY{n}{QuantumCircuit}\PY{p}{(}\PY{l+m+mi}{2}\PY{p}{)}
\PY{n}{bell}\PY{o}{.}\PY{n}{h}\PY{p}{(}\PY{l+m+mi}{0}\PY{p}{)}
\PY{n}{bell}\PY{o}{.}\PY{n}{cx}\PY{p}{(}\PY{l+m+mi}{0}\PY{p}{,} \PY{l+m+mi}{1}\PY{p}{)}
\PY{n}{bell}\PY{o}{.}\PY{n}{measure\PYZus{}all}\PY{p}{(}\PY{p}{)}

\PY{c+c1}{\PYZsh{} create two parameterized circuits}
\PY{n}{pqc} \PY{o}{=} \PY{n}{RealAmplitudes}\PY{p}{(}\PY{n}{num\PYZus{}qubits}\PY{o}{=}\PY{l+m+mi}{2}\PY{p}{,} \PY{n}{reps}\PY{o}{=}\PY{l+m+mi}{2}\PY{p}{)}
\PY{n}{pqc}\PY{o}{.}\PY{n}{measure\PYZus{}all}\PY{p}{(}\PY{p}{)}
\PY{n}{pqc} \PY{o}{=} \PY{n}{transpile}\PY{p}{(}\PY{n}{pqc}\PY{p}{,} \PY{n}{sim}\PY{p}{,} \PY{n}{optimization\PYZus{}level}\PY{o}{=}\PY{l+m+mi}{0}\PY{p}{)}
\PY{n}{pqc2} \PY{o}{=} \PY{n}{RealAmplitudes}\PY{p}{(}\PY{n}{num\PYZus{}qubits}\PY{o}{=}\PY{l+m+mi}{2}\PY{p}{,} \PY{n}{reps}\PY{o}{=}\PY{l+m+mi}{3}\PY{p}{)}
\PY{n}{pqc2}\PY{o}{.}\PY{n}{measure\PYZus{}all}\PY{p}{(}\PY{p}{)}
\PY{n}{pqc2} \PY{o}{=} \PY{n}{transpile}\PY{p}{(}\PY{n}{pqc2}\PY{p}{,} \PY{n}{sim}\PY{p}{,} \PY{n}{optimization\PYZus{}level}\PY{o}{=}\PY{l+m+mi}{0}\PY{p}{)}

\PY{n}{theta1} \PY{o}{=} \PY{p}{[}\PY{l+m+mi}{0}\PY{p}{,} \PY{l+m+mi}{1}\PY{p}{,} \PY{l+m+mi}{1}\PY{p}{,} \PY{l+m+mi}{2}\PY{p}{,} \PY{l+m+mi}{3}\PY{p}{,} \PY{l+m+mi}{5}\PY{p}{]}
\PY{n}{theta2} \PY{o}{=} \PY{p}{[}\PY{l+m+mi}{0}\PY{p}{,} \PY{l+m+mi}{1}\PY{p}{,} \PY{l+m+mi}{2}\PY{p}{,} \PY{l+m+mi}{3}\PY{p}{,} \PY{l+m+mi}{4}\PY{p}{,} \PY{l+m+mi}{5}\PY{p}{,} \PY{l+m+mi}{6}\PY{p}{,} \PY{l+m+mi}{7}\PY{p}{]}

\PY{c+c1}{\PYZsh{} initialization of the sampler}
\PY{n}{sampler} \PY{o}{=} \PY{n}{SamplerV2}\PY{p}{(}\PY{p}{)}

\PY{c+c1}{\PYZsh{} collect 128 shots from the Bell circuit}
\PY{n}{job} \PY{o}{=} \PY{n}{sampler}\PY{o}{.}\PY{n}{run}\PY{p}{(}\PY{p}{[}\PY{n}{bell}\PY{p}{]}\PY{p}{,} \PY{n}{shots}\PY{o}{=}\PY{l+m+mi}{128}\PY{p}{)}
\PY{n}{job\PYZus{}result} \PY{o}{=} \PY{n}{job}\PY{o}{.}\PY{n}{result}\PY{p}{(}\PY{p}{)}
\PY{n+nb}{print}\PY{p}{(}\PY{l+s+sa}{f}\PY{l+s+s2}{\PYZdq{}}\PY{l+s+s2}{counts for Bell circuit : }\PY{l+s+si}{\PYZob{}}\PY{n}{job\PYZus{}result}\PY{p}{[}\PY{l+m+mi}{0}\PY{p}{]}\PY{o}{.}\PY{n}{data}\PY{o}{.}\PY{n}{meas}\PY{o}{.}\PY{n}{get\PYZus{}counts}\PY{p}{(}\PY{p}{)}\PY{l+s+si}{\PYZcb{}}\PY{l+s+s2}{\PYZdq{}}\PY{p}{)}
 
\PY{c+c1}{\PYZsh{} run a sampler job on the parameterized circuits}
\PY{n}{job2} \PY{o}{=} \PY{n}{sampler}\PY{o}{.}\PY{n}{run}\PY{p}{(}\PY{p}{[}\PY{p}{(}\PY{n}{pqc}\PY{p}{,} \PY{n}{theta1}\PY{p}{)}\PY{p}{,} \PY{p}{(}\PY{n}{pqc2}\PY{p}{,} \PY{n}{theta2}\PY{p}{)}\PY{p}{]}\PY{p}{)}
\PY{n}{job\PYZus{}result} \PY{o}{=} \PY{n}{job2}\PY{o}{.}\PY{n}{result}\PY{p}{(}\PY{p}{)}
\PY{n+nb}{print}\PY{p}{(}\PY{l+s+sa}{f}\PY{l+s+s2}{\PYZdq{}}\PY{l+s+s2}{counts for parameterized circuit : }\PY{l+s+si}{\PYZob{}}\PY{n}{job\PYZus{}result}\PY{p}{[}\PY{l+m+mi}{0}\PY{p}{]}\PY{o}{.}\PY{n}{data}\PY{o}{.}\PY{n}{meas}\PY{o}{.}\PY{n}{get\PYZus{}counts}\PY{p}{(}\PY{p}{)}\PY{l+s+si}{\PYZcb{}}\PY{l+s+s2}{\PYZdq{}}\PY{p}{)}

\PY{c+c1}{\PYZsh{} \PYZhy{}\PYZhy{}\PYZhy{}\PYZhy{}\PYZhy{}\PYZhy{}\PYZhy{}\PYZhy{}\PYZhy{}\PYZhy{}\PYZhy{}\PYZhy{}\PYZhy{}\PYZhy{}\PYZhy{}\PYZhy{}\PYZhy{}\PYZhy{}\PYZhy{}\PYZhy{}\PYZhy{}\PYZhy{}\PYZhy{}\PYZhy{}\PYZhy{}\PYZhy{}\PYZhy{}\PYZhy{}\PYZhy{}\PYZhy{}\PYZhy{}\PYZhy{}\PYZhy{}\PYZhy{}\PYZhy{}\PYZhy{}\PYZhy{}\PYZhy{}\PYZhy{}\PYZhy{}\PYZhy{}\PYZhy{}\PYZhy{}\PYZhy{}\PYZhy{}\PYZhy{}\PYZhy{}\PYZhy{}\PYZhy{}\PYZhy{}}
\PY{c+c1}{\PYZsh{} Simulating with noise model from actual hardware}
\PY{c+c1}{\PYZsh{} \PYZhy{}\PYZhy{}\PYZhy{}\PYZhy{}\PYZhy{}\PYZhy{}\PYZhy{}\PYZhy{}\PYZhy{}\PYZhy{}\PYZhy{}\PYZhy{}\PYZhy{}\PYZhy{}\PYZhy{}\PYZhy{}\PYZhy{}\PYZhy{}\PYZhy{}\PYZhy{}\PYZhy{}\PYZhy{}\PYZhy{}\PYZhy{}\PYZhy{}\PYZhy{}\PYZhy{}\PYZhy{}\PYZhy{}\PYZhy{}\PYZhy{}\PYZhy{}\PYZhy{}\PYZhy{}\PYZhy{}\PYZhy{}\PYZhy{}\PYZhy{}\PYZhy{}\PYZhy{}\PYZhy{}\PYZhy{}\PYZhy{}\PYZhy{}\PYZhy{}\PYZhy{}\PYZhy{}\PYZhy{}\PYZhy{}\PYZhy{}}
\PY{k+kn}{from} \PY{n+nn}{qiskit\PYZus{}ibm\PYZus{}runtime} \PY{k+kn}{import} \PY{n}{QiskitRuntimeService}
\PY{n}{provider} \PY{o}{=} \PY{n}{QiskitRuntimeService}\PY{p}{(}\PY{p}{)} \PY{c+c1}{\PYZsh{}channel=\PYZsq{}ibm\PYZus{}quantum\PYZsq{}, token=\PYZdq{}set your own token here\PYZdq{})}
\PY{n}{backend} \PY{o}{=} \PY{n}{provider}\PY{o}{.}\PY{n}{get\PYZus{}backend}\PY{p}{(}\PY{l+s+s2}{\PYZdq{}}\PY{l+s+s2}{ibm\PYZus{}kyoto}\PY{l+s+s2}{\PYZdq{}}\PY{p}{)}

\PY{c+c1}{\PYZsh{} create sampler from the actual backend}
\PY{n}{sampler}\PY{o}{.}\PY{n}{from\PYZus{}backend}\PY{p}{(}\PY{n}{backend}\PY{p}{)}

\PY{c+c1}{\PYZsh{} run a sampler job on the parameterized circuits with noise model of the actual hardware}
\PY{n}{job3} \PY{o}{=} \PY{n}{sampler}\PY{o}{.}\PY{n}{run}\PY{p}{(}\PY{p}{[}\PY{p}{(}\PY{n}{pqc}\PY{p}{,} \PY{n}{theta1}\PY{p}{)}\PY{p}{,} \PY{p}{(}\PY{n}{pqc2}\PY{p}{,} \PY{n}{theta2}\PY{p}{)}\PY{p}{]}\PY{p}{)}
\PY{n}{job\PYZus{}result} \PY{o}{=} \PY{n}{job3}\PY{o}{.}\PY{n}{result}\PY{p}{(}\PY{p}{)}
\PY{n+nb}{print}\PY{p}{(}\PY{l+s+sa}{f}\PY{l+s+s2}{\PYZdq{}}\PY{l+s+s2}{Parameterized for Bell circuit w/noise: }\PY{l+s+si}{\PYZob{}}\PY{n}{job\PYZus{}result}\PY{p}{[}\PY{l+m+mi}{0}\PY{p}{]}\PY{o}{.}\PY{n}{data}\PY{o}{.}\PY{n}{meas}\PY{o}{.}\PY{n}{get\PYZus{}counts}\PY{p}{(}\PY{p}{)}\PY{l+s+si}{\PYZcb{}}\PY{l+s+s2}{\PYZdq{}}\PY{p}{)}
\end{Verbatim}
\end{tcolorbox}

    \begin{Verbatim}[commandchars=\\\{\}]
expectation values : psi1 = [ 1.5557232  -1.09625114], psi2 =
0.16332666909040922
counts for Bell circuit : \{'00': 71, '11': 57\}
counts for parameterized circuit : \{'01': 379, '00': 128, '10': 95, '11': 422\}
Parameterized for Bell circuit w/noise: \{'10': 103, '11': 423, '00': 107, '01':
391\}
    \end{Verbatim}


    % Add a bibliography block to the postdoc
    
    
     


\graphicspath{ {appendix_shor_circuit/} }
\section{Appendix: Shor's Algorithm}


\graphicspath{ {appendix_grover_circuit/} }
    \hypertarget{grovers-algorithm}{%
\section{Grover's Algorithm}\label{grovers-algorithm}}

Code is from IBM tutorial:
https://learning.quantum.ibm.com/tutorial/grovers-algorithm

\hypertarget{setup}{%
\subsection{Setup}\label{setup}}

On this run the backend was `ibm\_kyoto' which is a 27 qubit system.

    \begin{tcolorbox}[breakable, size=fbox, boxrule=1pt, pad at break*=1mm,colback=cellbackground, colframe=cellborder]
\prompt{In}{incolor}{1}{\boxspacing}
\begin{Verbatim}[commandchars=\\\{\}]
\PY{c+c1}{\PYZsh{} Built\PYZhy{}in modules}
\PY{k+kn}{import} \PY{n+nn}{math}

\PY{c+c1}{\PYZsh{} Imports from Qiskit}
\PY{k+kn}{from} \PY{n+nn}{qiskit} \PY{k+kn}{import} \PY{n}{QuantumCircuit}
\PY{k+kn}{from} \PY{n+nn}{qiskit}\PY{n+nn}{.}\PY{n+nn}{circuit}\PY{n+nn}{.}\PY{n+nn}{library} \PY{k+kn}{import} \PY{n}{GroverOperator}\PY{p}{,} \PY{n}{MCMT}\PY{p}{,} \PY{n}{ZGate}
\PY{k+kn}{from} \PY{n+nn}{qiskit}\PY{n+nn}{.}\PY{n+nn}{visualization} \PY{k+kn}{import} \PY{n}{plot\PYZus{}distribution}

\PY{c+c1}{\PYZsh{} Imports from Qiskit Runtime}
\PY{k+kn}{from} \PY{n+nn}{qiskit\PYZus{}ibm\PYZus{}runtime} \PY{k+kn}{import} \PY{n}{QiskitRuntimeService}
\PY{k+kn}{from} \PY{n+nn}{qiskit\PYZus{}ibm\PYZus{}runtime} \PY{k+kn}{import} \PY{n}{SamplerV2} \PY{k}{as} \PY{n}{Sampler}
\end{Verbatim}
\end{tcolorbox}

    \begin{tcolorbox}[breakable, size=fbox, boxrule=1pt, pad at break*=1mm,colback=cellbackground, colframe=cellborder]
\prompt{In}{incolor}{2}{\boxspacing}
\begin{Verbatim}[commandchars=\\\{\}]
\PY{c+c1}{\PYZsh{} Load saved credentials}
\PY{n}{service} \PY{o}{=} \PY{n}{QiskitRuntimeService}\PY{p}{(}\PY{p}{)}
\end{Verbatim}
\end{tcolorbox}

    \begin{tcolorbox}[breakable, size=fbox, boxrule=1pt, pad at break*=1mm,colback=cellbackground, colframe=cellborder]
\prompt{In}{incolor}{3}{\boxspacing}
\begin{Verbatim}[commandchars=\\\{\}]
\PY{c+c1}{\PYZsh{} To run on hardware, select the backend with the fewest number of jobs in the queue}
\PY{n}{service} \PY{o}{=} \PY{n}{QiskitRuntimeService}\PY{p}{(}\PY{n}{channel}\PY{o}{=}\PY{l+s+s2}{\PYZdq{}}\PY{l+s+s2}{ibm\PYZus{}quantum}\PY{l+s+s2}{\PYZdq{}}\PY{p}{)}
\PY{n}{backend} \PY{o}{=} \PY{n}{service}\PY{o}{.}\PY{n}{least\PYZus{}busy}\PY{p}{(}\PY{n}{operational}\PY{o}{=}\PY{k+kc}{True}\PY{p}{,} \PY{n}{simulator}\PY{o}{=}\PY{k+kc}{False}\PY{p}{)}
\PY{n}{backend}\PY{o}{.}\PY{n}{name}
\end{Verbatim}
\end{tcolorbox}

            \begin{tcolorbox}[breakable, size=fbox, boxrule=.5pt, pad at break*=1mm, opacityfill=0]
\prompt{Out}{outcolor}{3}{\boxspacing}
\begin{Verbatim}[commandchars=\\\{\}]
'ibm\_kyoto'
\end{Verbatim}
\end{tcolorbox}
        
    \hypertarget{map-classical-inputs-to-a-quantum-problem}{%
\subsection{Map classical inputs to a quantum
problem}\label{map-classical-inputs-to-a-quantum-problem}}

Grover's algorithm uses an oracle that marks computational basis states
with a phase of -1.

A controlled-Z gate marks the \(2^N − 1\) state (*NN bit-string). Here
the \texttt{MCMT}gate is used to implement the multi-controlled Z-gate.

    \begin{tcolorbox}[breakable, size=fbox, boxrule=1pt, pad at break*=1mm,colback=cellbackground, colframe=cellborder]
\prompt{In}{incolor}{4}{\boxspacing}
\begin{Verbatim}[commandchars=\\\{\}]
\PY{k}{def} \PY{n+nf}{grover\PYZus{}oracle}\PY{p}{(}\PY{n}{marked\PYZus{}states}\PY{p}{)}\PY{p}{:}
\PY{+w}{    }\PY{l+s+sd}{\PYZdq{}\PYZdq{}\PYZdq{}Build a Grover oracle for multiple marked states}

\PY{l+s+sd}{    Here we assume all input marked states have the same number of bits}

\PY{l+s+sd}{    Parameters:}
\PY{l+s+sd}{        marked\PYZus{}states (str or list): Marked states of oracle}

\PY{l+s+sd}{    Returns:}
\PY{l+s+sd}{        QuantumCircuit: Quantum circuit representing Grover oracle}
\PY{l+s+sd}{    \PYZdq{}\PYZdq{}\PYZdq{}}
    \PY{k}{if} \PY{o+ow}{not} \PY{n+nb}{isinstance}\PY{p}{(}\PY{n}{marked\PYZus{}states}\PY{p}{,} \PY{n+nb}{list}\PY{p}{)}\PY{p}{:}
        \PY{n}{marked\PYZus{}states} \PY{o}{=} \PY{p}{[}\PY{n}{marked\PYZus{}states}\PY{p}{]}
    \PY{c+c1}{\PYZsh{} Compute the number of qubits in circuit}
    \PY{n}{num\PYZus{}qubits} \PY{o}{=} \PY{n+nb}{len}\PY{p}{(}\PY{n}{marked\PYZus{}states}\PY{p}{[}\PY{l+m+mi}{0}\PY{p}{]}\PY{p}{)}

    \PY{n}{qc} \PY{o}{=} \PY{n}{QuantumCircuit}\PY{p}{(}\PY{n}{num\PYZus{}qubits}\PY{p}{)}
    \PY{c+c1}{\PYZsh{} Mark each target state in the input list}
    \PY{k}{for} \PY{n}{target} \PY{o+ow}{in} \PY{n}{marked\PYZus{}states}\PY{p}{:}
        \PY{c+c1}{\PYZsh{} Flip target bit\PYZhy{}string to match Qiskit bit\PYZhy{}ordering}
        \PY{n}{rev\PYZus{}target} \PY{o}{=} \PY{n}{target}\PY{p}{[}\PY{p}{:}\PY{p}{:}\PY{o}{\PYZhy{}}\PY{l+m+mi}{1}\PY{p}{]}
        \PY{c+c1}{\PYZsh{} Find the indices of all the \PYZsq{}0\PYZsq{} elements in bit\PYZhy{}string}
        \PY{n}{zero\PYZus{}inds} \PY{o}{=} \PY{p}{[}\PY{n}{ind} \PY{k}{for} \PY{n}{ind} \PY{o+ow}{in} \PY{n+nb}{range}\PY{p}{(}\PY{n}{num\PYZus{}qubits}\PY{p}{)} \PY{k}{if} \PY{n}{rev\PYZus{}target}\PY{o}{.}\PY{n}{startswith}\PY{p}{(}\PY{l+s+s2}{\PYZdq{}}\PY{l+s+s2}{0}\PY{l+s+s2}{\PYZdq{}}\PY{p}{,} \PY{n}{ind}\PY{p}{)}\PY{p}{]}
        \PY{c+c1}{\PYZsh{} Add a multi\PYZhy{}controlled Z\PYZhy{}gate with pre\PYZhy{} and post\PYZhy{}applied X\PYZhy{}gates (open\PYZhy{}controls)}
        \PY{c+c1}{\PYZsh{} where the target bit\PYZhy{}string has a \PYZsq{}0\PYZsq{} entry}
        \PY{n}{qc}\PY{o}{.}\PY{n}{x}\PY{p}{(}\PY{n}{zero\PYZus{}inds}\PY{p}{)}
        \PY{n}{qc}\PY{o}{.}\PY{n}{compose}\PY{p}{(}\PY{n}{MCMT}\PY{p}{(}\PY{n}{ZGate}\PY{p}{(}\PY{p}{)}\PY{p}{,} \PY{n}{num\PYZus{}qubits} \PY{o}{\PYZhy{}} \PY{l+m+mi}{1}\PY{p}{,} \PY{l+m+mi}{1}\PY{p}{)}\PY{p}{,} \PY{n}{inplace}\PY{o}{=}\PY{k+kc}{True}\PY{p}{)}
        \PY{n}{qc}\PY{o}{.}\PY{n}{x}\PY{p}{(}\PY{n}{zero\PYZus{}inds}\PY{p}{)}
    \PY{k}{return} \PY{n}{qc}
\end{Verbatim}
\end{tcolorbox}

    Using the oracle function we can define a specific instance of Grover
search.

In this example we will mark two computational states out of the eight
available in a three-qubit computational space:

    \begin{tcolorbox}[breakable, size=fbox, boxrule=1pt, pad at break*=1mm,colback=cellbackground, colframe=cellborder]
\prompt{In}{incolor}{5}{\boxspacing}
\begin{Verbatim}[commandchars=\\\{\}]
\PY{n}{marked\PYZus{}states} \PY{o}{=} \PY{p}{[}\PY{l+s+s2}{\PYZdq{}}\PY{l+s+s2}{011}\PY{l+s+s2}{\PYZdq{}}\PY{p}{,} \PY{l+s+s2}{\PYZdq{}}\PY{l+s+s2}{100}\PY{l+s+s2}{\PYZdq{}}\PY{p}{]}

\PY{n}{oracle} \PY{o}{=} \PY{n}{grover\PYZus{}oracle}\PY{p}{(}\PY{n}{marked\PYZus{}states}\PY{p}{)}
\PY{n}{oracle}\PY{o}{.}\PY{n}{draw}\PY{p}{(}\PY{n}{output}\PY{o}{=}\PY{l+s+s2}{\PYZdq{}}\PY{l+s+s2}{mpl}\PY{l+s+s2}{\PYZdq{}}\PY{p}{,} \PY{n}{style}\PY{o}{=}\PY{l+s+s2}{\PYZdq{}}\PY{l+s+s2}{iqp}\PY{l+s+s2}{\PYZdq{}}\PY{p}{)}
\end{Verbatim}
\end{tcolorbox}
 
            
\prompt{Out}{outcolor}{5}{}
    
    \begin{center}
    \adjustimage{max size={0.9\linewidth}{0.9\paperheight}}{output_7_0.png}
    \end{center}
    { \hspace*{\fill} \\}
    

    The built-in Qiskit \texttt{GroverOperator} takes an oracle circuit and
returns a circuit that is composed of the oracle circuit itself and a
circuit that amplifies the states marked by the oracle.

    \begin{tcolorbox}[breakable, size=fbox, boxrule=1pt, pad at break*=1mm,colback=cellbackground, colframe=cellborder]
\prompt{In}{incolor}{6}{\boxspacing}
\begin{Verbatim}[commandchars=\\\{\}]
\PY{n}{grover\PYZus{}op} \PY{o}{=} \PY{n}{GroverOperator}\PY{p}{(}\PY{n}{oracle}\PY{p}{)}
\PY{n}{grover\PYZus{}op}\PY{o}{.}\PY{n}{decompose}\PY{p}{(}\PY{p}{)}\PY{o}{.}\PY{n}{draw}\PY{p}{(}\PY{n}{output}\PY{o}{=}\PY{l+s+s2}{\PYZdq{}}\PY{l+s+s2}{mpl}\PY{l+s+s2}{\PYZdq{}}\PY{p}{,} \PY{n}{style}\PY{o}{=}\PY{l+s+s2}{\PYZdq{}}\PY{l+s+s2}{iqp}\PY{l+s+s2}{\PYZdq{}}\PY{p}{)}
\end{Verbatim}
\end{tcolorbox}
 
            
\prompt{Out}{outcolor}{6}{}
    
    \begin{center}
    \adjustimage{max size={0.9\linewidth}{0.9\paperheight}}{output_9_0.png}
    \end{center}
    { \hspace*{\fill} \\}
    

    Repeated operations amplify the marked states:

    \begin{tcolorbox}[breakable, size=fbox, boxrule=1pt, pad at break*=1mm,colback=cellbackground, colframe=cellborder]
\prompt{In}{incolor}{7}{\boxspacing}
\begin{Verbatim}[commandchars=\\\{\}]
\PY{n}{optimal\PYZus{}num\PYZus{}iterations} \PY{o}{=} \PY{n}{math}\PY{o}{.}\PY{n}{floor}\PY{p}{(}
    \PY{n}{math}\PY{o}{.}\PY{n}{pi} \PY{o}{/} \PY{p}{(}\PY{l+m+mi}{4} \PY{o}{*} \PY{n}{math}\PY{o}{.}\PY{n}{asin}\PY{p}{(}\PY{n}{math}\PY{o}{.}\PY{n}{sqrt}\PY{p}{(}\PY{n+nb}{len}\PY{p}{(}\PY{n}{marked\PYZus{}states}\PY{p}{)} \PY{o}{/} \PY{l+m+mi}{2}\PY{o}{*}\PY{o}{*}\PY{n}{grover\PYZus{}op}\PY{o}{.}\PY{n}{num\PYZus{}qubits}\PY{p}{)}\PY{p}{)}\PY{p}{)}
\PY{p}{)}
\end{Verbatim}
\end{tcolorbox}

    \hypertarget{full-circuit}{%
\subsubsection{Full circuit}\label{full-circuit}}

We use the \texttt{qc.h()} function to put a Hadamard gate on each
qubit, creating the even superposition of all computational basis
states. The Grover operator is applied the optimal number of times using
the \texttt{QuantumCircuit.power(...)} method.

    \begin{tcolorbox}[breakable, size=fbox, boxrule=1pt, pad at break*=1mm,colback=cellbackground, colframe=cellborder]
\prompt{In}{incolor}{8}{\boxspacing}
\begin{Verbatim}[commandchars=\\\{\}]
\PY{n}{qc} \PY{o}{=} \PY{n}{QuantumCircuit}\PY{p}{(}\PY{n}{grover\PYZus{}op}\PY{o}{.}\PY{n}{num\PYZus{}qubits}\PY{p}{)}
\PY{c+c1}{\PYZsh{} Create even superposition of all basis states}
\PY{n}{qc}\PY{o}{.}\PY{n}{h}\PY{p}{(}\PY{n+nb}{range}\PY{p}{(}\PY{n}{grover\PYZus{}op}\PY{o}{.}\PY{n}{num\PYZus{}qubits}\PY{p}{)}\PY{p}{)}
\PY{c+c1}{\PYZsh{} Apply Grover operator the optimal number of times}
\PY{n}{qc}\PY{o}{.}\PY{n}{compose}\PY{p}{(}\PY{n}{grover\PYZus{}op}\PY{o}{.}\PY{n}{power}\PY{p}{(}\PY{n}{optimal\PYZus{}num\PYZus{}iterations}\PY{p}{)}\PY{p}{,} \PY{n}{inplace}\PY{o}{=}\PY{k+kc}{True}\PY{p}{)}
\PY{c+c1}{\PYZsh{} Measure all qubits}
\PY{n}{qc}\PY{o}{.}\PY{n}{measure\PYZus{}all}\PY{p}{(}\PY{p}{)}
\PY{n}{qc}\PY{o}{.}\PY{n}{draw}\PY{p}{(}\PY{n}{output}\PY{o}{=}\PY{l+s+s2}{\PYZdq{}}\PY{l+s+s2}{mpl}\PY{l+s+s2}{\PYZdq{}}\PY{p}{,} \PY{n}{style}\PY{o}{=}\PY{l+s+s2}{\PYZdq{}}\PY{l+s+s2}{iqp}\PY{l+s+s2}{\PYZdq{}}\PY{p}{)}
\end{Verbatim}
\end{tcolorbox}
 
            
\prompt{Out}{outcolor}{8}{}
    
    \begin{center}
    \adjustimage{max size={0.9\linewidth}{0.9\paperheight}}{output_13_0.png}
    \end{center}
    { \hspace*{\fill} \\}
    

    \hypertarget{optimise-the-problem-for-quantum-execution}{%
\subsection{Optimise the problem for quantum
execution}\label{optimise-the-problem-for-quantum-execution}}

    \begin{tcolorbox}[breakable, size=fbox, boxrule=1pt, pad at break*=1mm,colback=cellbackground, colframe=cellborder]
\prompt{In}{incolor}{9}{\boxspacing}
\begin{Verbatim}[commandchars=\\\{\}]
\PY{k+kn}{from} \PY{n+nn}{qiskit}\PY{n+nn}{.}\PY{n+nn}{transpiler}\PY{n+nn}{.}\PY{n+nn}{preset\PYZus{}passmanagers} \PY{k+kn}{import} \PY{n}{generate\PYZus{}preset\PYZus{}pass\PYZus{}manager}

\PY{n}{target} \PY{o}{=} \PY{n}{backend}\PY{o}{.}\PY{n}{target}
\PY{n}{pm} \PY{o}{=} \PY{n}{generate\PYZus{}preset\PYZus{}pass\PYZus{}manager}\PY{p}{(}\PY{n}{target}\PY{o}{=}\PY{n}{target}\PY{p}{,} \PY{n}{optimization\PYZus{}level}\PY{o}{=}\PY{l+m+mi}{3}\PY{p}{)}

\PY{n}{circuit\PYZus{}isa} \PY{o}{=} \PY{n}{pm}\PY{o}{.}\PY{n}{run}\PY{p}{(}\PY{n}{qc}\PY{p}{)}
\PY{n}{circuit\PYZus{}isa}\PY{o}{.}\PY{n}{draw}\PY{p}{(}\PY{n}{output}\PY{o}{=}\PY{l+s+s2}{\PYZdq{}}\PY{l+s+s2}{mpl}\PY{l+s+s2}{\PYZdq{}}\PY{p}{,} \PY{n}{idle\PYZus{}wires}\PY{o}{=}\PY{k+kc}{False}\PY{p}{,} \PY{n}{style}\PY{o}{=}\PY{l+s+s2}{\PYZdq{}}\PY{l+s+s2}{iqp}\PY{l+s+s2}{\PYZdq{}}\PY{p}{)}
\end{Verbatim}
\end{tcolorbox}
 
            
\prompt{Out}{outcolor}{9}{}
    
    \begin{center}
    \adjustimage{max size={0.9\linewidth}{0.9\paperheight}}{output_15_0.png}
    \end{center}
    { \hspace*{\fill} \\}
    

    \hypertarget{execute}{%
\subsection{Execute}\label{execute}}

Amplitude amplification is a sampling problem that is suitable for
execution with the \texttt{Sampler} runtime primitive.

    \begin{tcolorbox}[breakable, size=fbox, boxrule=1pt, pad at break*=1mm,colback=cellbackground, colframe=cellborder]
\prompt{In}{incolor}{10}{\boxspacing}
\begin{Verbatim}[commandchars=\\\{\}]
\PY{c+c1}{\PYZsh{} To run on local simulator:}
\PY{c+c1}{\PYZsh{}   1. Use the SatetvectorSampler from qiskit.primitives instead}
\PY{n}{sampler} \PY{o}{=} \PY{n}{Sampler}\PY{p}{(}\PY{n}{backend}\PY{o}{=}\PY{n}{backend}\PY{p}{)}
\PY{n}{sampler}\PY{o}{.}\PY{n}{options}\PY{o}{.}\PY{n}{default\PYZus{}shots} \PY{o}{=} \PY{l+m+mi}{10\PYZus{}000}
\PY{n}{result} \PY{o}{=} \PY{n}{sampler}\PY{o}{.}\PY{n}{run}\PY{p}{(}\PY{p}{[}\PY{n}{circuit\PYZus{}isa}\PY{p}{]}\PY{p}{)}\PY{o}{.}\PY{n}{result}\PY{p}{(}\PY{p}{)}
\PY{n}{dist} \PY{o}{=} \PY{n}{result}\PY{p}{[}\PY{l+m+mi}{0}\PY{p}{]}\PY{o}{.}\PY{n}{data}\PY{o}{.}\PY{n}{meas}\PY{o}{.}\PY{n}{get\PYZus{}counts}\PY{p}{(}\PY{p}{)}
\end{Verbatim}
\end{tcolorbox}

    \hypertarget{post-process}{%
\subsection{Post process}\label{post-process}}

    \begin{tcolorbox}[breakable, size=fbox, boxrule=1pt, pad at break*=1mm,colback=cellbackground, colframe=cellborder]
\prompt{In}{incolor}{11}{\boxspacing}
\begin{Verbatim}[commandchars=\\\{\}]
\PY{n}{plot\PYZus{}distribution}\PY{p}{(}\PY{n}{dist}\PY{p}{)}
\end{Verbatim}
\end{tcolorbox}
 
            
\prompt{Out}{outcolor}{11}{}
    
    \begin{center}
    \adjustimage{max size={0.9\linewidth}{0.9\paperheight}}{output_19_0.png}
    \end{center}
    { \hspace*{\fill} \\}
    



\end{document}

%%% Local Variables:
%%% mode: latex
%%% TeX-master: t
%%% End:
