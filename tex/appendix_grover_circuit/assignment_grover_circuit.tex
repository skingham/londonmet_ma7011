    \hypertarget{grovers-algorithm}{%
\section{Grover's Algorithm}\label{grovers-algorithm}}

Code is from IBM tutorial:
https://learning.quantum.ibm.com/tutorial/grovers-algorithm

\hypertarget{setup}{%
\subsection{Setup}\label{setup}}

On this run the backend was `ibm\_kyoto' which is a 27 qubit system.

    \begin{tcolorbox}[breakable, size=fbox, boxrule=1pt, pad at break*=1mm,colback=cellbackground, colframe=cellborder]
\prompt{In}{incolor}{1}{\boxspacing}
\begin{Verbatim}[commandchars=\\\{\}]
\PY{c+c1}{\PYZsh{} Built\PYZhy{}in modules}
\PY{k+kn}{import} \PY{n+nn}{math}

\PY{c+c1}{\PYZsh{} Imports from Qiskit}
\PY{k+kn}{from} \PY{n+nn}{qiskit} \PY{k+kn}{import} \PY{n}{QuantumCircuit}
\PY{k+kn}{from} \PY{n+nn}{qiskit}\PY{n+nn}{.}\PY{n+nn}{circuit}\PY{n+nn}{.}\PY{n+nn}{library} \PY{k+kn}{import} \PY{n}{GroverOperator}\PY{p}{,} \PY{n}{MCMT}\PY{p}{,} \PY{n}{ZGate}
\PY{k+kn}{from} \PY{n+nn}{qiskit}\PY{n+nn}{.}\PY{n+nn}{visualization} \PY{k+kn}{import} \PY{n}{plot\PYZus{}distribution}

\PY{c+c1}{\PYZsh{} Imports from Qiskit Runtime}
\PY{k+kn}{from} \PY{n+nn}{qiskit\PYZus{}ibm\PYZus{}runtime} \PY{k+kn}{import} \PY{n}{QiskitRuntimeService}
\PY{k+kn}{from} \PY{n+nn}{qiskit\PYZus{}ibm\PYZus{}runtime} \PY{k+kn}{import} \PY{n}{SamplerV2} \PY{k}{as} \PY{n}{Sampler}
\end{Verbatim}
\end{tcolorbox}

    \begin{tcolorbox}[breakable, size=fbox, boxrule=1pt, pad at break*=1mm,colback=cellbackground, colframe=cellborder]
\prompt{In}{incolor}{2}{\boxspacing}
\begin{Verbatim}[commandchars=\\\{\}]
\PY{c+c1}{\PYZsh{} Load saved credentials}
\PY{n}{service} \PY{o}{=} \PY{n}{QiskitRuntimeService}\PY{p}{(}\PY{p}{)}
\end{Verbatim}
\end{tcolorbox}

    \begin{tcolorbox}[breakable, size=fbox, boxrule=1pt, pad at break*=1mm,colback=cellbackground, colframe=cellborder]
\prompt{In}{incolor}{3}{\boxspacing}
\begin{Verbatim}[commandchars=\\\{\}]
\PY{c+c1}{\PYZsh{} To run on hardware, select the backend with the fewest number of jobs in the queue}
\PY{n}{service} \PY{o}{=} \PY{n}{QiskitRuntimeService}\PY{p}{(}\PY{n}{channel}\PY{o}{=}\PY{l+s+s2}{\PYZdq{}}\PY{l+s+s2}{ibm\PYZus{}quantum}\PY{l+s+s2}{\PYZdq{}}\PY{p}{)}
\PY{n}{backend} \PY{o}{=} \PY{n}{service}\PY{o}{.}\PY{n}{least\PYZus{}busy}\PY{p}{(}\PY{n}{operational}\PY{o}{=}\PY{k+kc}{True}\PY{p}{,} \PY{n}{simulator}\PY{o}{=}\PY{k+kc}{False}\PY{p}{)}
\PY{n}{backend}\PY{o}{.}\PY{n}{name}
\end{Verbatim}
\end{tcolorbox}

            \begin{tcolorbox}[breakable, size=fbox, boxrule=.5pt, pad at break*=1mm, opacityfill=0]
\prompt{Out}{outcolor}{3}{\boxspacing}
\begin{Verbatim}[commandchars=\\\{\}]
'ibm\_kyoto'
\end{Verbatim}
\end{tcolorbox}
        
    \hypertarget{map-classical-inputs-to-a-quantum-problem}{%
\subsection{Map classical inputs to a quantum
problem}\label{map-classical-inputs-to-a-quantum-problem}}

Grover's algorithm uses an oracle that marks computational basis states
with a phase of -1.

A controlled-Z gate marks the \(2^N − 1\) state (*NN bit-string). Here
the \texttt{MCMT}gate is used to implement the multi-controlled Z-gate.

    \begin{tcolorbox}[breakable, size=fbox, boxrule=1pt, pad at break*=1mm,colback=cellbackground, colframe=cellborder]
\prompt{In}{incolor}{4}{\boxspacing}
\begin{Verbatim}[commandchars=\\\{\}]
\PY{k}{def} \PY{n+nf}{grover\PYZus{}oracle}\PY{p}{(}\PY{n}{marked\PYZus{}states}\PY{p}{)}\PY{p}{:}
\PY{+w}{    }\PY{l+s+sd}{\PYZdq{}\PYZdq{}\PYZdq{}Build a Grover oracle for multiple marked states}

\PY{l+s+sd}{    Here we assume all input marked states have the same number of bits}

\PY{l+s+sd}{    Parameters:}
\PY{l+s+sd}{        marked\PYZus{}states (str or list): Marked states of oracle}

\PY{l+s+sd}{    Returns:}
\PY{l+s+sd}{        QuantumCircuit: Quantum circuit representing Grover oracle}
\PY{l+s+sd}{    \PYZdq{}\PYZdq{}\PYZdq{}}
    \PY{k}{if} \PY{o+ow}{not} \PY{n+nb}{isinstance}\PY{p}{(}\PY{n}{marked\PYZus{}states}\PY{p}{,} \PY{n+nb}{list}\PY{p}{)}\PY{p}{:}
        \PY{n}{marked\PYZus{}states} \PY{o}{=} \PY{p}{[}\PY{n}{marked\PYZus{}states}\PY{p}{]}
    \PY{c+c1}{\PYZsh{} Compute the number of qubits in circuit}
    \PY{n}{num\PYZus{}qubits} \PY{o}{=} \PY{n+nb}{len}\PY{p}{(}\PY{n}{marked\PYZus{}states}\PY{p}{[}\PY{l+m+mi}{0}\PY{p}{]}\PY{p}{)}

    \PY{n}{qc} \PY{o}{=} \PY{n}{QuantumCircuit}\PY{p}{(}\PY{n}{num\PYZus{}qubits}\PY{p}{)}
    \PY{c+c1}{\PYZsh{} Mark each target state in the input list}
    \PY{k}{for} \PY{n}{target} \PY{o+ow}{in} \PY{n}{marked\PYZus{}states}\PY{p}{:}
        \PY{c+c1}{\PYZsh{} Flip target bit\PYZhy{}string to match Qiskit bit\PYZhy{}ordering}
        \PY{n}{rev\PYZus{}target} \PY{o}{=} \PY{n}{target}\PY{p}{[}\PY{p}{:}\PY{p}{:}\PY{o}{\PYZhy{}}\PY{l+m+mi}{1}\PY{p}{]}
        \PY{c+c1}{\PYZsh{} Find the indices of all the \PYZsq{}0\PYZsq{} elements in bit\PYZhy{}string}
        \PY{n}{zero\PYZus{}inds} \PY{o}{=} \PY{p}{[}\PY{n}{ind} \PY{k}{for} \PY{n}{ind} \PY{o+ow}{in} \PY{n+nb}{range}\PY{p}{(}\PY{n}{num\PYZus{}qubits}\PY{p}{)} \PY{k}{if} \PY{n}{rev\PYZus{}target}\PY{o}{.}\PY{n}{startswith}\PY{p}{(}\PY{l+s+s2}{\PYZdq{}}\PY{l+s+s2}{0}\PY{l+s+s2}{\PYZdq{}}\PY{p}{,} \PY{n}{ind}\PY{p}{)}\PY{p}{]}
        \PY{c+c1}{\PYZsh{} Add a multi\PYZhy{}controlled Z\PYZhy{}gate with pre\PYZhy{} and post\PYZhy{}applied X\PYZhy{}gates (open\PYZhy{}controls)}
        \PY{c+c1}{\PYZsh{} where the target bit\PYZhy{}string has a \PYZsq{}0\PYZsq{} entry}
        \PY{n}{qc}\PY{o}{.}\PY{n}{x}\PY{p}{(}\PY{n}{zero\PYZus{}inds}\PY{p}{)}
        \PY{n}{qc}\PY{o}{.}\PY{n}{compose}\PY{p}{(}\PY{n}{MCMT}\PY{p}{(}\PY{n}{ZGate}\PY{p}{(}\PY{p}{)}\PY{p}{,} \PY{n}{num\PYZus{}qubits} \PY{o}{\PYZhy{}} \PY{l+m+mi}{1}\PY{p}{,} \PY{l+m+mi}{1}\PY{p}{)}\PY{p}{,} \PY{n}{inplace}\PY{o}{=}\PY{k+kc}{True}\PY{p}{)}
        \PY{n}{qc}\PY{o}{.}\PY{n}{x}\PY{p}{(}\PY{n}{zero\PYZus{}inds}\PY{p}{)}
    \PY{k}{return} \PY{n}{qc}
\end{Verbatim}
\end{tcolorbox}

    Using the oracle function we can define a specific instance of Grover
search.

In this example we will mark two computational states out of the eight
available in a three-qubit computational space:

    \begin{tcolorbox}[breakable, size=fbox, boxrule=1pt, pad at break*=1mm,colback=cellbackground, colframe=cellborder]
\prompt{In}{incolor}{5}{\boxspacing}
\begin{Verbatim}[commandchars=\\\{\}]
\PY{n}{marked\PYZus{}states} \PY{o}{=} \PY{p}{[}\PY{l+s+s2}{\PYZdq{}}\PY{l+s+s2}{011}\PY{l+s+s2}{\PYZdq{}}\PY{p}{,} \PY{l+s+s2}{\PYZdq{}}\PY{l+s+s2}{100}\PY{l+s+s2}{\PYZdq{}}\PY{p}{]}

\PY{n}{oracle} \PY{o}{=} \PY{n}{grover\PYZus{}oracle}\PY{p}{(}\PY{n}{marked\PYZus{}states}\PY{p}{)}
\PY{n}{oracle}\PY{o}{.}\PY{n}{draw}\PY{p}{(}\PY{n}{output}\PY{o}{=}\PY{l+s+s2}{\PYZdq{}}\PY{l+s+s2}{mpl}\PY{l+s+s2}{\PYZdq{}}\PY{p}{,} \PY{n}{style}\PY{o}{=}\PY{l+s+s2}{\PYZdq{}}\PY{l+s+s2}{iqp}\PY{l+s+s2}{\PYZdq{}}\PY{p}{)}
\end{Verbatim}
\end{tcolorbox}
 
            
\prompt{Out}{outcolor}{5}{}
    
    \begin{center}
    \adjustimage{max size={0.9\linewidth}{0.9\paperheight}}{output_7_0.png}
    \end{center}
    { \hspace*{\fill} \\}
    

    The built-in Qiskit \texttt{GroverOperator} takes an oracle circuit and
returns a circuit that is composed of the oracle circuit itself and a
circuit that amplifies the states marked by the oracle.

    \begin{tcolorbox}[breakable, size=fbox, boxrule=1pt, pad at break*=1mm,colback=cellbackground, colframe=cellborder]
\prompt{In}{incolor}{6}{\boxspacing}
\begin{Verbatim}[commandchars=\\\{\}]
\PY{n}{grover\PYZus{}op} \PY{o}{=} \PY{n}{GroverOperator}\PY{p}{(}\PY{n}{oracle}\PY{p}{)}
\PY{n}{grover\PYZus{}op}\PY{o}{.}\PY{n}{decompose}\PY{p}{(}\PY{p}{)}\PY{o}{.}\PY{n}{draw}\PY{p}{(}\PY{n}{output}\PY{o}{=}\PY{l+s+s2}{\PYZdq{}}\PY{l+s+s2}{mpl}\PY{l+s+s2}{\PYZdq{}}\PY{p}{,} \PY{n}{style}\PY{o}{=}\PY{l+s+s2}{\PYZdq{}}\PY{l+s+s2}{iqp}\PY{l+s+s2}{\PYZdq{}}\PY{p}{)}
\end{Verbatim}
\end{tcolorbox}
 
            
\prompt{Out}{outcolor}{6}{}
    
    \begin{center}
    \adjustimage{max size={0.9\linewidth}{0.9\paperheight}}{output_9_0.png}
    \end{center}
    { \hspace*{\fill} \\}
    

    Repeated operations amplify the marked states:

    \begin{tcolorbox}[breakable, size=fbox, boxrule=1pt, pad at break*=1mm,colback=cellbackground, colframe=cellborder]
\prompt{In}{incolor}{7}{\boxspacing}
\begin{Verbatim}[commandchars=\\\{\}]
\PY{n}{optimal\PYZus{}num\PYZus{}iterations} \PY{o}{=} \PY{n}{math}\PY{o}{.}\PY{n}{floor}\PY{p}{(}
    \PY{n}{math}\PY{o}{.}\PY{n}{pi} \PY{o}{/} \PY{p}{(}\PY{l+m+mi}{4} \PY{o}{*} \PY{n}{math}\PY{o}{.}\PY{n}{asin}\PY{p}{(}\PY{n}{math}\PY{o}{.}\PY{n}{sqrt}\PY{p}{(}\PY{n+nb}{len}\PY{p}{(}\PY{n}{marked\PYZus{}states}\PY{p}{)} \PY{o}{/} \PY{l+m+mi}{2}\PY{o}{*}\PY{o}{*}\PY{n}{grover\PYZus{}op}\PY{o}{.}\PY{n}{num\PYZus{}qubits}\PY{p}{)}\PY{p}{)}\PY{p}{)}
\PY{p}{)}
\end{Verbatim}
\end{tcolorbox}

    \hypertarget{full-circuit}{%
\subsubsection{Full circuit}\label{full-circuit}}

We use the \texttt{qc.h()} function to put a Hadamard gate on each
qubit, creating the even superposition of all computational basis
states. The Grover operator is applied the optimal number of times using
the \texttt{QuantumCircuit.power(...)} method.

    \begin{tcolorbox}[breakable, size=fbox, boxrule=1pt, pad at break*=1mm,colback=cellbackground, colframe=cellborder]
\prompt{In}{incolor}{8}{\boxspacing}
\begin{Verbatim}[commandchars=\\\{\}]
\PY{n}{qc} \PY{o}{=} \PY{n}{QuantumCircuit}\PY{p}{(}\PY{n}{grover\PYZus{}op}\PY{o}{.}\PY{n}{num\PYZus{}qubits}\PY{p}{)}
\PY{c+c1}{\PYZsh{} Create even superposition of all basis states}
\PY{n}{qc}\PY{o}{.}\PY{n}{h}\PY{p}{(}\PY{n+nb}{range}\PY{p}{(}\PY{n}{grover\PYZus{}op}\PY{o}{.}\PY{n}{num\PYZus{}qubits}\PY{p}{)}\PY{p}{)}
\PY{c+c1}{\PYZsh{} Apply Grover operator the optimal number of times}
\PY{n}{qc}\PY{o}{.}\PY{n}{compose}\PY{p}{(}\PY{n}{grover\PYZus{}op}\PY{o}{.}\PY{n}{power}\PY{p}{(}\PY{n}{optimal\PYZus{}num\PYZus{}iterations}\PY{p}{)}\PY{p}{,} \PY{n}{inplace}\PY{o}{=}\PY{k+kc}{True}\PY{p}{)}
\PY{c+c1}{\PYZsh{} Measure all qubits}
\PY{n}{qc}\PY{o}{.}\PY{n}{measure\PYZus{}all}\PY{p}{(}\PY{p}{)}
\PY{n}{qc}\PY{o}{.}\PY{n}{draw}\PY{p}{(}\PY{n}{output}\PY{o}{=}\PY{l+s+s2}{\PYZdq{}}\PY{l+s+s2}{mpl}\PY{l+s+s2}{\PYZdq{}}\PY{p}{,} \PY{n}{style}\PY{o}{=}\PY{l+s+s2}{\PYZdq{}}\PY{l+s+s2}{iqp}\PY{l+s+s2}{\PYZdq{}}\PY{p}{)}
\end{Verbatim}
\end{tcolorbox}
 
            
\prompt{Out}{outcolor}{8}{}
    
    \begin{center}
    \adjustimage{max size={0.9\linewidth}{0.9\paperheight}}{output_13_0.png}
    \end{center}
    { \hspace*{\fill} \\}
    

    \hypertarget{optimise-the-problem-for-quantum-execution}{%
\subsection{Optimise the problem for quantum
execution}\label{optimise-the-problem-for-quantum-execution}}

    \begin{tcolorbox}[breakable, size=fbox, boxrule=1pt, pad at break*=1mm,colback=cellbackground, colframe=cellborder]
\prompt{In}{incolor}{9}{\boxspacing}
\begin{Verbatim}[commandchars=\\\{\}]
\PY{k+kn}{from} \PY{n+nn}{qiskit}\PY{n+nn}{.}\PY{n+nn}{transpiler}\PY{n+nn}{.}\PY{n+nn}{preset\PYZus{}passmanagers} \PY{k+kn}{import} \PY{n}{generate\PYZus{}preset\PYZus{}pass\PYZus{}manager}

\PY{n}{target} \PY{o}{=} \PY{n}{backend}\PY{o}{.}\PY{n}{target}
\PY{n}{pm} \PY{o}{=} \PY{n}{generate\PYZus{}preset\PYZus{}pass\PYZus{}manager}\PY{p}{(}\PY{n}{target}\PY{o}{=}\PY{n}{target}\PY{p}{,} \PY{n}{optimization\PYZus{}level}\PY{o}{=}\PY{l+m+mi}{3}\PY{p}{)}

\PY{n}{circuit\PYZus{}isa} \PY{o}{=} \PY{n}{pm}\PY{o}{.}\PY{n}{run}\PY{p}{(}\PY{n}{qc}\PY{p}{)}
\PY{n}{circuit\PYZus{}isa}\PY{o}{.}\PY{n}{draw}\PY{p}{(}\PY{n}{output}\PY{o}{=}\PY{l+s+s2}{\PYZdq{}}\PY{l+s+s2}{mpl}\PY{l+s+s2}{\PYZdq{}}\PY{p}{,} \PY{n}{idle\PYZus{}wires}\PY{o}{=}\PY{k+kc}{False}\PY{p}{,} \PY{n}{style}\PY{o}{=}\PY{l+s+s2}{\PYZdq{}}\PY{l+s+s2}{iqp}\PY{l+s+s2}{\PYZdq{}}\PY{p}{)}
\end{Verbatim}
\end{tcolorbox}
 
            
\prompt{Out}{outcolor}{9}{}
    
    \begin{center}
    \adjustimage{max size={0.9\linewidth}{0.9\paperheight}}{output_15_0.png}
    \end{center}
    { \hspace*{\fill} \\}
    

    \hypertarget{execute}{%
\subsection{Execute}\label{execute}}

Amplitude amplification is a sampling problem that is suitable for
execution with the \texttt{Sampler} runtime primitive.

    \begin{tcolorbox}[breakable, size=fbox, boxrule=1pt, pad at break*=1mm,colback=cellbackground, colframe=cellborder]
\prompt{In}{incolor}{10}{\boxspacing}
\begin{Verbatim}[commandchars=\\\{\}]
\PY{c+c1}{\PYZsh{} To run on local simulator:}
\PY{c+c1}{\PYZsh{}   1. Use the SatetvectorSampler from qiskit.primitives instead}
\PY{n}{sampler} \PY{o}{=} \PY{n}{Sampler}\PY{p}{(}\PY{n}{backend}\PY{o}{=}\PY{n}{backend}\PY{p}{)}
\PY{n}{sampler}\PY{o}{.}\PY{n}{options}\PY{o}{.}\PY{n}{default\PYZus{}shots} \PY{o}{=} \PY{l+m+mi}{10\PYZus{}000}
\PY{n}{result} \PY{o}{=} \PY{n}{sampler}\PY{o}{.}\PY{n}{run}\PY{p}{(}\PY{p}{[}\PY{n}{circuit\PYZus{}isa}\PY{p}{]}\PY{p}{)}\PY{o}{.}\PY{n}{result}\PY{p}{(}\PY{p}{)}
\PY{n}{dist} \PY{o}{=} \PY{n}{result}\PY{p}{[}\PY{l+m+mi}{0}\PY{p}{]}\PY{o}{.}\PY{n}{data}\PY{o}{.}\PY{n}{meas}\PY{o}{.}\PY{n}{get\PYZus{}counts}\PY{p}{(}\PY{p}{)}
\end{Verbatim}
\end{tcolorbox}

    \hypertarget{post-process}{%
\subsection{Post process}\label{post-process}}

    \begin{tcolorbox}[breakable, size=fbox, boxrule=1pt, pad at break*=1mm,colback=cellbackground, colframe=cellborder]
\prompt{In}{incolor}{11}{\boxspacing}
\begin{Verbatim}[commandchars=\\\{\}]
\PY{n}{plot\PYZus{}distribution}\PY{p}{(}\PY{n}{dist}\PY{p}{)}
\end{Verbatim}
\end{tcolorbox}
 
            
\prompt{Out}{outcolor}{11}{}
    
    \begin{center}
    \adjustimage{max size={0.9\linewidth}{0.9\paperheight}}{output_19_0.png}
    \end{center}
    { \hspace*{\fill} \\}
    
