%%%%%%%%%%%%%%%%%%%%%%%%%%%%%%%%%%%%%%%%%%%%%%%%%%%%%%%%%%%%%%%%%%%%%%%%%%%%%%%%%%%%%%%%%%%%%%%%%%%%%%%%%%%%%%%%%%%%%%%%
\section{QISKIT Quantum Circuit Implementations}

The IBM Quantum Platform was made available to the public in May 2016 \cite{IBM:2016b} and was composed of five
superconducting qubits and was housed at the IBM T.J. Watson Research Center in New York.

They make available real quantum computer resources, such as the recently release 'IBM Quantum Eagle' 127-qubit processor
and their 133-qubit Heron processor, with the desire to help users to \enquote{push the boundaries of more complex problems}.

The material in this section used the \emph{IBM Quantum Learning} course \emph{Phase Estimation and factoring} \cite{Qiskit:2024a} as a guide.


\subsection{Constructing Circuits \& the Quantum Fourier Transform}

The SDK has evolved over time and offers a rich set of tools to construct quantum circuits and platforms to run circuits on
real hardware, along with simulated systems that are calibrated to real systems, and can be used to simulate idea and noisy
quantum machines.  There is far to much in these SDKs to do more than scratch the surface.

The workflow we will be using is primarily:

\begin{enumerate}
\item Construct quantum circuits to represent gates, operators and algorithms.
\item Perform analysis on these circuits to check and understand the quantum states.
\item Provision backend services, either real or simulated, with particular properties such as qubit size and noise/error
  characteristics.
\item Run our quantum algorithm.
\item Measure results and report on the outcome of the experiments.
\end{enumerate}

\subsection{Quantum Fourier Transform}

All code for this section is in Appendix B.

As we have already seen, the \emph{Quantum Fourier Transform} (QFT) is a key building block for many algorithms, including
Shor's algorithm for matrix inversion.  We have seen the single qubit Hadamard gate to bring a qubit into an equal superposition
of state.  The second gate we need is a two-qubit controlled rotation \emph{$CROT_k$} to perform the phase shift.

We can manually build up the circuit by calling methods on the circuit object, building gates and connections:

%caption=First stage of the QFT circiut,[language=python]
\singlespace
%\begin{lstlisting}
\inputminted{python}{tex/code/qft-stage-1.py}
\label{listing:py1}
%\end{lstlisting}
\doublespacing

\begin{figure}[H] % Single column figure
   \includegraphics[width=0.35\textwidth]{appendix_qiskit_intro/output_3_0.png}\hfill
   \caption{First stage of the QFT circuit.}
   \label{fig:qft-stage-1}
\end{figure}

\subsubsection{Encoding Inputs onto a Circuit}

To set an input value we need to encode the value $5$ onto the qubits.  The Hadamard gate leaves all states equally likely.
This can be accomplished by applying the required phase shift to each qubit with a command like
\texttt{circuit.p(5*pi/4,0)} and we encode the state vector $ | \tilde{5} \rangle $:

\begin{figure}[H] % Single column figure
   \includegraphics[width=0.30\textwidth]{appendix_qiskit_intro/output_4_0.png}\hfill
   \caption{Qubits in Fourier state $ | \tilde{5} \rangle $. }
   \label{fig:fourier-state-1}
\end{figure}

We can visulise this as a Bloch state vector for our three qubits:

\begin{figure}[H] % Single column figure
   \includegraphics[width=0.95\textwidth]{appendix_qiskit_intro/output_6_0.png}\hfill
   \caption{Bloch state vector for $ | \tilde{5} \rangle $.}
   \label{fig:bloch-1}
\end{figure}

\subsubsection{Using Pretested Quantum Circuit Components}

It is not necessary to construct a QFT circuit by hand.  We can construct QFT and inverse QFT components
using classes defined within the SDK more simply and with less chance of errors:

\singlespace
\begin{minted}
  {python}
from qiskit.circuit.library import QFT
qft = QFT(num_qubits=3, approximation_degree=0, do_swaps=True,
          inverse=True, insert_barriers=False, name='qft').to_gate()
circuit.append(qft)
\end{minted}
\doublespacing

\subsubsection{Running the Circuit}

The task we want to perform is to run the inverted QFT algo and recover the value $5$ from the system.

For a three cubit circuit we can simulate this on the 5-qubit system \texttt{qiskit\_ibm\_runtime.fake\_provider import FakeManilaV2}.

We transpile our circuit into something that our selected backend will run, then execute the algorithm:

\singlespace
\begin{minted}
  {python}
backend = FakeManilaV2()
shots = 2048
transpiled_qc = transpile(circuit, backend, optimization_level=3)
job = backend.run(transpiled_qc, shots=shots)

counts = job.result().get_counts()
plot_histogram(counts)
\end{minted}
\doublespacing

\begin{figure}[H] % Single column figure
   \includegraphics[width=0.95\textwidth]{appendix_qiskit_intro/output_8_0.png}\hfill
   \includegraphics[width=0.95\textwidth]{appendix_qiskit_intro/output_9_0.png}\hfill
   \caption{Full inverse QFT circuit and measurement histogram.}
   \label{fig:iqft-hist-1}
\end{figure}

And we see we recover the value $101$.


\subsection{Grover}

All code and results for this section is in Appendix C.

For the Grover algorithm, from the QISKIT `qiskit.circuit.library` we have a number of gates and operators that simplify the construction
of the algorithm.  What we are left to do is to build is the oracle to mark the required states (details in appendix).

In this example we have a 3-qubit system, and select two states - \texttt{['011', '100']} for the oracle to mark.  This is done using the
multi-controlled multi-target gate `MCMT`.  The given routine `grover\_operation` generates the following quantum circuit where the marked states
can be seen encoded as inverted binary values on the first and last stage: 

\begin{figure}[H] % Single column figure
   \includegraphics[width=0.45\textwidth]{appendix_grover_circuit/output_7_0.png}\hfill
   %\includesvg[width=0.95\textwidth]{Lattice-reduction}\hfill
   \caption{Grover oracle circuit.}
   \label{fig:grover-opertor}
\end{figure}

We calculate the optimal number of iterations and QISKIT primitives can then knit our grover operator into the following gate
construction:

\begin{figure}[H] % Single column figure
   \includegraphics[width=0.95\textwidth]{appendix_grover_circuit/output_15_0.png}\hfill
   %\includesvg[width=0.95\textwidth]{Lattice-reduction}\hfill
   \caption{Final Grover circuit with repeated Grover operators.}
   \label{fig:grover-gate}
\end{figure}

Looking only at the Hadamard gate initialisers, and the completed Grover gate as one diagram element, we can see the quantum measurements
instrumentation:

\begin{figure}[H] % Single column figure
   \includegraphics[width=0.95\textwidth]{appendix_grover_circuit/output_13_0.png}\hfill
   \includegraphics[width=0.95\textwidth]{appendix_grover_circuit/output_19_0.png}\hfill
   %\includesvg[width=0.95\textwidth]{Lattice-reduction}\hfill
   \caption{High level circuit and resultant measurement histogram.}
   \label{fig:grover-histogram}
\end{figure}

After running the algorithm we find, as expected, the marked states appearing with the highest probabilities.

