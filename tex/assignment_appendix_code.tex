\section{Appendix: Setting up IBM QISKIT SDK}

\subsection{SDK Versions and Environment Setup}

 For this paper we are using the \href{https://www.ibm.com/quantum/qiskit}{Qiskit SDK} \cite{Qiskit:2023} version \texttt{1.1.0}.   Our quantum state diagrams are generated using the \href{https://qutip.org}{QuTip} 
 \emph{Quantum Toolbox in Python} project.  And we will generally want an environment that allows us to run example code in a \href{https://jupyter.org}{Jupyter} notebook.  It is also assumed that 
 a full \texttt{Jupyter} and a \LaTeX environment have been provisioned.
 
 Using \href{https://conda.io/projects/conda/en/latest/user-guide/getting-started.html}{Conda}, our environment is created thus:

\begin{listing}[!ht]
\inputminted{bash}{tex/code/install_qiskit_conda.sh}
\caption{Setting up the QISKIT conda environment}
\label{listing:1}
\end{listing}

Pip can also be used in an appropriate virtual environment:

\begin{listing}[!ht]
\inputminted{bash}{tex/code/install_qiskit_pip.sh}
\caption{Setting up the QISKIT venv/pip environment}
\label{listing:2}
\end{listing}

If we only run a simulator locally we don't need to setup an account with the \emph{IBM Quantum Platform} or \emph{IBM Cloud}.  We have registered for the \textbf{Open plan} and so we setup
access to an \href(https://docs.quantum.ibm.com/start/setup-channel){IBM Quantum Channel}.  We can get an \emph{API token} from our IBM Quantum Platform profile page and test the 
connection with the following code:

\begin{listing}[!ht]
\inputminted{python}{tex/code/setup_ibm_service_channel.py}
\caption{Test connection to the Qiskit Runtime Service}
\label{listing:3}
\end{listing}

\subsection{Quantum Diagrams and Principals}


\subsection{Shor's Algorithm}


\subsection{Grover's Algorithm}




