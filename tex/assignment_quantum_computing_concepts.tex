%%%%%%%%%%%%%%%%%%%%%%%%%%%%%%%%%%%%%%%%%%%%%%%%%%%%%%%%%%%%%%%%%%%%%%%%%%%%%%%%%%%%%%%%%%%%%%%%%%%%%%%%%%%%%%%%%%%%%%%%
\section{Quantum Computing Concepts}

The fundamental building block of the is a \emph{qubit}.   Qubits can be constructed as [get details of IBM cubit], trapped-ion, photonics, silicon-based, and others

Logical cubits can be made from more than one physical qubits [\ldots] to provide error correction.

\emph{Quantum gates} \href{https://en.wikipedia.org/wiki/Quantum_logic_gate}{wiki} are reversible unitary transformations on at least one qubit \ldots

The model we use for quantum computation is  an \emph{acyclic quantum circuit} or \emph{quantum gate arrays}, where this computation is a sequence of quantum gates, measurements and initialisation of qubits to known values
\href{https://en.wikipedia.org/wiki/Quantum_circuit}{wiki}

\emph{Decoherence}


?The average T1 time of a qubit is the time it takes for a qubit to decay from the excited state to the ground state. It is important because it limits the duration of meaningful programs we can run on the quantum computer.,? stated in Qiskit Textbook (IBM, 2021b) within the section titled ?Calibrating Qubits with Qiskit Pulse.?

Notation: we use Paul Dirac's bra-ket notation to 

\subsection{Quantum Algorithms}

\subsubsection{Shor}

When it was published in \citeyear{Shor:1997} there were no actual quantum computers in existence, but
Peter Shor's paper \citetitle{Shor:1997} \cite{Shor:1997},
Following on from Bernstein and Vazirani [1993] and  Simon [1994], who an oracle problem which takes
polynomial time on a quantum computer but requires exponential time on a classical computer  ushered into
practical quantum algorithms using a \emph{quantum gate array} to construct subroutines that were used
in his algorithms to of reversible modular exponentiation, quantum Fourier transforms, prime factorisation
and solving discrete algorithms.  The last two showing the practical application of the hypothetical
quantum computer to break the, now traditional, RSA \cite{Rivest:1978}, and Diffie-Hellman \cite{Diffie:1976}
\& ElGamal \cite{ElGamal:1985} public-key cryptographic schemes.

\subsubsection{Grover's Algorithm}

\subsubsection{GEECM}



