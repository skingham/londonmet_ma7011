%%%%%%%%%%%%%%%%%%%%%%%%%%%%%%%%%%%%%%%%%%%%%%%%%%%%%%%%%%%%%%%%%%%%%%%%%%%%%%%%%%%%%%%%%%%%%%%%%%%%%%%%%%%%%%%%%%%%%%%%
\section{Introduction}
When Peter Shor published his original paper in \citeyear{Shor:1994} \cite{Shor:1994} and then his expanded paper,
\citetitle{Shor:1994} in \citeyear{Shor:1997} \cite{Shor:1997}, not only were there no actual
quantum computers in existence, but quantum computing research was in a nascent state.  Shor himself details the
development of the idea of using quantum mechanics for computation in the early 1980s by Benioff [\ldots cite please]
and the suggestion by Feynman [\ldots cite please] that quantum mechanics might be computationally more powerful
than Turing machines.  

A brace of papers in the early 1990s by Deutsche and Jozsa [\ldots cite please] and Berthiaume and Brassard
[\ldots cite please] went further showing that a quantum computer could solve exactly and quickly a class of 
problems, \emph{bounded error probability probabilistic polynomial time} (BPP),  that classical computers can solve
quickly only with high probability using randomness \cite{Babai:1993}. 

Shor's papers went substantially further by taking inspiration from from Simon in 1994 [\ldots cite], who refined an
oracle problem and who demonstrated a quantum solution in polynomial time that requires exponential time on a classical
computer.  Shor tackled two number theory problems that underpin the Diffie-Helman \cite{Diffie:1976}, RSA
\cite{Rivest:1978}, and ElGamal \cite{ElGamal:1985} public key encryption and digital signature schemes; namely quantum
algorithms for finding discrete logarithms and factor integers.

Following on from Shor, \citeauthor{Grover:1996} \citeyear{Grover:1996} \cite{Grover:1996} proved that quantum computers
could perform an unstructured search with a quadratic speedup over a classical computing search
($O(\sqrt{N})$ vs $O(N)$.)  Whilst not reducing an exponential problem to a polynomial time problem, it does allow the
practical brute-force applications.
 
Because of the importance of both of these quantum circuits in breaking classical encryption schemes they are also
important to understand from a practical vantage point.  These quantum algorithms provide a floor from which new
research can be expected to find better and faster quantum algorithms.  Having exposure to these quantum circuit
ideas should also help in understanding why new \emph{post-quantum cryptography} (PQC) schemes from NIST
\cite{NIST:2022} are currently thought to be immune to quantum computer attacks.

Actual quantum computers are now a reality, and will become more powerful each year and the technology advances.
Although thereare a plethora of quantum simulators available \cite{Dargan:2022}, both IBM \cite{IBM:2024a} and Google
\cite{Google:2024a} have software development kits (SDKs) for python that allow quantum circuits to be designed and
run on real quantum computing resources and quantum simulators.  This paper is seeks to demonstrate how to implement
Shor's and Grover's algorithms using IBM's \emph{Quantum Information Science Kit} (QISKIT) SDK.

The rest of the paper is organised as follows. Section 2 is an overview of the quantum computing concepts needed to
understand the SDK and the implementation of the two circuits.  Section 3 develops the actual code that runs on IBMs
systems and shows the results, demonstrating the speedups and the practical limitations on the current generation of
quantum computers.  Section 4 is a review of the current mathematical cryptographic techniques and why they are
thought to be immune to being broken by quantum computing.  Section 5 is the conclusion and my thoughts on what I
have learnt writing this report.


\begin{quote}
?By using quantum mechanics in a computer can you compute more efficiently than on a classical computer?? The first to ask this question explicitly was Deutsch [1985, 1989]. In order to study this question, he defined both quantum Turing machines and quantum circuits and investigated some of their properties.

The question of whether using quantum mechanics in a computer allows one to obtain more computational power was more recently addressed by Deutsch and Jozsa [1992] and Berthiaume and Brassard [1992, 1994]. These papers showed that there are problems which quantum computers can quickly solve exactly, but that classical computers can only solve quickly with high probability and the aid of a random number generator.

However, these papers did not show how to solve any problem in quantum polynomial time that was not already known to be solvable in polynomial time with the aid of a random number generator, allowing a small probability of error; this is the characterization of the complexity class BPP (bounded error probability probabilistic polynomial time), which is widely viewed as the class of efficiently solvable problems.

Further work on this problem was stimulated by Bernstein and Vazirani [1993]. One of the results contained in their paper was an oracle problem (that is, a problem involving a ?black box? subroutine that the computer is allowed to perform but for which no code is accessible) which can be done in polynomial time on a quantum Turing machine but which requires superpolynomial time on a classical computer. This result was improved by Simon [1994], who gave a much simpler construction of an oracle problem which takes polynomial time on a quantum computer but requires exponential time on a classical computer. Indeed, while Bernstein and Vaziarni?s problem appears contrived, Simon?s problem looks quite natural. Simon?s algorithm inspired the work presented in this paper.
\end{quote}

% \begin{figure}[!ht] % Single column figure
%   %\includegraphics[width=0.95\textwidth]{statistic_id267132_annual-amount-of-financial-damage-caused-by-reported-cybercrime-in-us-2001-2022.png}\hfill
%   \includesvg[width=0.95\textwidth]{Lattice-reduction}\hfill
%   \caption{Latice Reduction \autocite{Wikipedia:2021}  Source: Wikipedia.}
%   \label{fig:lattice-reduction}
% \end{figur\subsection{Post-Quantum Cryptography}
