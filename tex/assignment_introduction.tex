%%%%%%%%%%%%%%%%%%%%%%%%%%%%%%%%%%%%%%%%%%%%%%%%%%%%%%%%%%%%%%%%%%%%%%%%%%%%%%%%%%%%%%%%%%%%%%%%%%%%%%%%%%%%%%%%%%%%%%%%
\section{Introduction}
When Peter Shor published his original paper in \citeyear{Shor:1994} \cite{Shor:1994} -- and then his expanded paper 
\citetitle{Shor:1994} in \citeyear{Shor:1997} \cite{Shor:1997} -- not only were there no actual
quantum computers in existence, but quantum computing research was in a nascent state.  Shor himself details the
development of the idea of using quantum mechanics for computation in the early 1980s by Benioff \cite{Benioff:1982}
and the suggestion by Feynman \cite{Feynman:1986} that quantum mechanics might be computationally more powerful
than Turing machines.  

A brace of papers in the early 1990s by Deutsch and Jozsa \cite{Deutsch:1992} and Berthiaume and Brassard
\cite{Berthiaume:1992} went further showing that a quantum computer could solve exactly and quickly a class of 
problems -- \emph{bounded error probability probabilistic polynomial time} (BPP) problems  -- that classical computers
can solve quickly only with high probability by using randomness \cite{Babai:1993}, and demonstrated this using the
\emph{oracle problem}.

Shor's papers went substantially further by taking inspiration from from Simon's 1994 paper \citetitle{Simon:1994}
\cite{Simon:1994} which refined the oracle problem and who demonstrated a quantum solution in polynomial time that
would require exponential time on a classical computer.  Shor tackled two number theory problems that underpin the
Diffie-Helman \cite{Diffie:1976}, RSA \cite{Rivest:1978}, and ElGamal \cite{ElGamal:1985} public key encryption
and digital signature schemes; namely quantum algorithms for finding discrete logarithms and factoring integers.

Following on from Shor, \citeauthor{Grover:1996} \citeyear{Grover:1996} \cite{Grover:1996} proved that quantum computers
could perform an unstructured search with a quadratic speedup over a classical computing search
($O(\sqrt{N})$ vs $O(N)$.)  Whilst not reducing an exponential problem to a polynomial time problem, it still does
allow some practical brute-force attacks.

Because of the importance of both of these quantum circuits in breaking classical encryption schemes it is
important to understand their construction from a practical vantage point.  These quantum algorithms provide a floor
from which new research can be expected to find better and faster quantum algorithms.  Having exposure to these
quantum circuit ideas should also help in understanding why new \emph{post-quantum cryptography} (PQC) schemes from
NIST \cite{NIST:2022} are currently thought to be immune to quantum computer attacks.

Current quantum computers, while not at the point of quantum supremacy, will become more powerful each year as the
technology advances.  These advances are being made available to researchers and students through various quantum
computing simulators \cite{Dargan:2022}, and  both IBM \cite{IBM:2024a} and Google
\cite{Google:2024a} have software development kits (SDKs) for python that allow quantum circuits to be designed and
run on real quantum computing hardware and quantum simulators.  This paper is seeks to demonstrate how to implement
Shor's and Grover's algorithms using IBM's \emph{Quantum Information Science Kit} (QISKIT) SDK.

The rest of the paper is organised as follows. Section 2 is an overview of the quantum computing concepts needed to
understand the SDK and the implementation of the two circuits.  Section 3 develops the actual code that runs on IBMs
systems and shows the results, demonstrating the speedups and the practical limitations on the current generation of
quantum computers.  Section 4 is a review of the current mathematical cryptographic techniques and why they are
thought to be immune to being broken by quantum computing.  Section 5 is the conclusion and my thoughts on what I
have learnt writing this report.


% \begin{figure}[!ht] % Single column figure
%   %\includegraphics[width=0.95\textwidth]{statistic_id267132_annual-amount-of-financial-damage-caused-by-reported-cybercrime-in-us-2001-2022.png}\hfill
%   \includesvg[width=0.95\textwidth]{Lattice-reduction}\hfill
%   \caption{Latice Reduction \autocite{Wikipedia:2021}  Source: Wikipedia.}
%   \label{fig:lattice-reduction}
% \end{figur\subsection{Post-Quantum Cryptography}
