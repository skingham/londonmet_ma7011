\section{Post-Quantum Cryptography}

Quantum systems can only break asymmetric cryptographic ciphers that, at their core, have mathematical problems that
appear to have exponential, or worse, complexity under classical computing, but
that can be solved more efficiently using quantum circuits algorithms.  The breaking of RSA and ElGamal
schemes by Shor's algorithms is instructive, as it uses a particular class of quantum gate, the QFT, to
speed up known algorithms to factor integers and solve discrete logarithm problems.

When NIST announced \cite{NIST:2022} in \citeyear{NIST:2022} the group of four PQC specifications selected, for general
encryption this was the CRYSTALS-Kyber algorithm; selected in part for the comparatively small encryption keys that two
parties can exchange easily and its speed of operation.  And for digital signatures NIST selected the three algorithms:
CRYSTALS-Dilithium, FALCON and SPHINCS+.

Of these, three of the algorithms are based on lattice schemes, and SPHINCS+ was selected, in part, as it
was not based on lattices, but hashing, as a backup.  NIST still is accepting new algorithms and we can
expect new standards to be added as new techniques arise.

In the \citeyear{Yalamuri:2022} paper \citetitle{Yalamuri:2022}, \citeauthor{Yalamuri:2022} identify six families of crypto-systems
that have attributes that appear at present to have an assumption of being hard for anticipated
quantum computers.

In this section we give the briefest review of these families, and some reasons for there being a lack
of efficient quantum algorithms that would break these schemes.

\subsection{Lattice-based Cryptography}

Lattice schemes have well researched problems such as the \emph{Shortest Vector Problem} (SVP) and
\emph{Learning With Errors} (LWE) known to be difficult.  These problems have the attribute that
solving the average-case instance is as hard as solving for the worst case.

The problems also involve high-dimensional geometric structures for which current quantum algorithms
struggle with.  These rich mathematical structures, along with the introduction of small random
errors in LWE for example, gives weight to the belief that they are robust against attacks.

\subsection{Code-based Cryptography}

Using the knowledge that the McEliece scheme remains unbroken, schemes bases on the difficulty of
decoding randomly generated linear codes - especially Goppa codes - are seen as a promising
bases for new schemes that have smaller key sizes than the McEliece system.

The general decoding problem for linear codes is believed to be hard for both classical and
quantum computers.

\subsection{Hash-based Cryptography}

Hash based schemes have minimal security assumptions, and have shown provable security properties
in studies.  Having a long history in digital signature protocols, there are promising
candidates for PQC.

\subsection{Multivariate Cryptography}

The authors of the study point out that although many multivariate codes have been broken, many
have not.  The hardness of solving multivariate quadratic equations over finite fields is known
to be NP-hard.  There are also a variety of existent schemes - Rainbow, Unbalanced Oil and vinegar (UOV),
Hidden Field Equations (HFE), Multivariate Quadrant (MQ) problems - and this provides a rich
area of research to improve key sizes to make these codes more practical.

\subsection{Isogeny-based Cryptography}

Although computational complexity is a problem for Isogenies-based schemes, this class relies
on the difficulty of computing isogenies between elliptic curves, especially super-singular isogeny
graphs where there is significant research.

The security of these schemes are based on well studied mathematical problems, but there is
a question whether smaller devices can cope with the computational requirements.

\subsection{Symmetric-key-based Cryptography}

For AES, and other similar symmetric schemes, the vulnerability to brute force attacks using Grover's algorithm
can be mitigated by increasing the key length.  The attractiveness of using such schemes for PQC is
tempered by the fact that they do not yet allow the properties of digital signing and non-repudiation that
asymmetric schemes give.

