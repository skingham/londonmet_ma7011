\section{Conclusion}

Using Shor's paper as a guide, I have explored practical aspects of the construction of quantum gates and their
composition into quantum circuits.  To do this I used the IBM QISKIT SDK using locally run quantum machinery
simulators to understand how to construct circuits on current quantum computers.  Although the platform does
allow access to real quantum computing resources, it was not necessary for the ambitions of this paper.


\subsection{Personal Insight}

Personally I found reading Shor's papers from the mid-1990's \cite{Shor:1994} \cite{Shor:1997} to be enlightening.  In
short order he was able to layout the salient points from a large and complex branch of science to neatly encapsulate
the mathematical methods and thinking needed to tackle the construction of a class of quantum algorithms.  The
success of his exposition is demonstrated by the brevity of Grover's paper, clocking in at only 8 pages \cite{Grover:1996}.

I had dual problems trying to write a comprehensible paper that both explains the concepts and could demonstrate the
principles on modern hardware simulators.

The first is that background mathematics is deep enough that some level
of detail is needed to understand the quantum gate operations and the representation of the physical system in the
matrix operations.  Many tutorials I have read as background for this paper struggle with this comprehension issue also.

The second problem is that the use of the SDK, whilst being somewhat intuitive, has undergone significant changes as
the real systems have developed.  This has meant that may tutorials are out of date, and that some concepts for
constructing circuits is not highlighted well, increasing the burden of bringing comprehensible examples to illustrate
Shor's and Grover's original work.

Although I have not achieved the outcome I has originally aimed for, I hope that this paper gives some insight into
the possibilities of integrating quantum computing projects into cryptographic studies.
